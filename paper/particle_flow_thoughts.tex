\documentclass{article}

% Graphics
\usepackage{graphicx}
\usepackage[caption=false,font=footnotesize]{subfig}

% Formatting
\usepackage[nodisplayskipstretch]{setspace} \onehalfspacing
\usepackage{color}
\usepackage{amsmath}
\usepackage{amsfonts}
\usepackage{bbm}

\usepackage[scaled]{helvet}
\renewcommand*\familydefault{\sfdefault} %% Only if the base font of the document is to be sans serif
\usepackage[T1]{fontenc}

% Environments
\usepackage{IEEEtrantools}
\usepackage{algorithm}
\usepackage{algorithmic}

% Logic
\usepackage{ifthen}
\usepackage{etoolbox}

% References
\usepackage{natbib}

% Drawing
\usepackage{tikz}
\usepackage{pgfplots}
 \usetikzlibrary{plotmarks}
 \pgfplotsset{compat=newest}
 \pgfplotsset{plot coordinates/math parser=false}
 \usepgfplotslibrary{external}
 \tikzexternalize[prefix=tikz/]

 %%% Macros %%%
%%% Theorem environments %%%
\newtheorem{theorem}{Theorem}[section]
\newtheorem{lemma}[theorem]{Lemma}
\newtheorem{proposition}[theorem]{Proposition}
\newtheorem{corollary}[theorem]{Corollary}
\newtheorem{model}[theorem]{Model}

\newenvironment{proof}[1][Proof]{\begin{trivlist}
\item[\hskip \labelsep {\bfseries #1}]}{\end{trivlist}}
\newenvironment{definition}[1][Definition]{\begin{trivlist}
\item[\hskip \labelsep {\bfseries #1}]}{\end{trivlist}}
\newenvironment{example}[1][Example]{\begin{trivlist}
\item[\hskip \labelsep {\bfseries #1}]}{\end{trivlist}}
\newenvironment{remark}[1][Remark]{\begin{trivlist}
\item[\hskip \labelsep {\bfseries #1}]}{\end{trivlist}}

\newcommand{\qed}{\nobreak \ifvmode \relax \else
      \ifdim\lastskip<1.5em \hskip-\lastskip
      \hskip1.5em plus0em minus0.5em \fi \nobreak
      \vrule height0.75em width0.5em depth0.25em\fi}
%%%%%%%%%%%%%%%%%%%%%%%%%%%%%%


\newcommand{\real}{\mathbb{R}}

\newcommand{\lhood}{l}
\newcommand{\nconst}[1]{K_{#1}}

\newcommand{\lsspace}{\mathcal{X}}
\newcommand{\lsdim}{{d_{\ls{}}}}

\newcommand{\obspace}{\mathcal{Y}}
\newcommand{\obdim}{{d_{\ob{}}}}

\newcommand{\priorden}{p}
\newcommand{\postden}{\pi}
\newcommand{\seqden}[1]{\pi_{#1}}
\newcommand{\seqdenapprox}[1]{\hat{\pi}_{#1}}
\newcommand{\impden}{q}
\newcommand{\incimpden}[1]{q_{#1}}

\newcommand{\logprior}{M}
\newcommand{\loglhood}{L}
\newcommand{\logseqden}[1]{\Xi_{#1}}
\newcommand{\logseqdenapprox}[1]{\hat{\Xi}_{#1}}

\newcommand{\sn}[1]{z_{#1}}
\newcommand{\snchange}[1]{\xi_{#1}}

\newcommand{\lsmnapprox}[1]{\hat{m}_{#1}}
\newcommand{\lsvrapprox}[1]{\hat{P}_{#1}}

\newcommand{\errconst}[1]{C_{#1}}



% Particle flow
\newcommand{\flowbm}[1]{\epsilon_{#1}}          % Particle flow Brownian motion
\newcommand{\flowdrift}[1]{\zeta_{#1}}          % Particle flow drift
\newcommand{\flowdiffuse}[1]{\eta_{#1}}         % Particle flow diffusion
\newcommand{\flowcov}[1]{D_{#1}}                % Particle flow "Covariance" matrix
\newcommand{\dsf}{\gamma}                % Particle flow diffusion scale factor

\newcommand{\lgmomapprox}[1]{\hat{\lgmom}_{#1}}       % Linear observation matrix formed by differentiation of the observation function
\newcommand{\obapprox}[1]{\hat{y}_{#1}}

\newcommand{\flowdriftapprox}[1]{\hat{\zeta}_{#1}}      % Approximate (actual) particle flow drift
\newcommand{\flowdiffuseapprox}[1]{\hat{\eta}_{#1}}     % Approximate (actual) particle flow diffu






%%% Basic maths stuff %%%
\newcommand{\zmatrix}{\mathsf{0}}
\newcommand{\idmatrix}{\mathsf{I}}
\newcommand{\half}{\frac{1}{2}}

%%% Functions and operators %%%
\newcommand{\minv}{^{-1}}
\newcommand{\determ}[1]{\left|#1\right|}
\newcommand{\magn}[1]{\left|#1\right|}
\newcommand{\pd}[3]{\left.\frac{\partial #1}{\partial #2}\right|_{#3}}
\newcommand{\npd}[4]{\left.\frac{\partial^{#1} #2}{\partial #3}\right|_{#4}}
\newcommand{\expect}[1]{\mathbb{E}_{#1}}
\newcommand{\variance}[1]{\mathbb{V}_{#1}}
\newcommand{\bigo}[1]{\mathcal{O}\left(#1\right)}
\newcommand{\prob}{P}
\newcommand{\indic}[1]{\mathbbm{1}_{#1}}
\DeclareMathOperator{\sinc}{sinc}
\DeclareMathOperator{\trace}{Tr}
\newcommand{\degr}{^{\circ}}
\newcommand{\rightasconverge}{\stackrel{a.s.}{\rightarrow}}
\newcommand{\leftasconverge}{\stackrel{a.s.}{\leftarrow}}

%%% Probability spaces %%%
\newcommand{\reals}{\mathbb{R}}

%%% Time, indexes, etc. %%%
\newcommand{\ti}{n}
\newcommand{\dct}[1]{t_{#1}}
\newcommand{\timax}{N}

%%% States, observations, etc. %%%
\newcommand{\ls}[1]{x_{#1}}
\newcommand{\ob}[1]{y_{#1}}
\newcommand{\rls}[1]{u_{#1}}
\newcommand{\mls}[1]{z_{#1}}

%%% Distributions %%%
\newcommand{\normalden}[3]{\mathcal{N}\left(#1\left|\vphantom{#1}#2,#3\right.\right)}
\newcommand{\uniformden}[2]{\mathcal{U}\left(\left[#1,#2\right]\right)}
\newcommand{\dirac}[1]{\delta_{#1}}
\newcommand{\gammaden}[3]{\mathcal{G}\left(#1\left|#2,#3\right.\right)}
\newcommand{\studenttden}[4]{\mathcal{ST}\left(#1|#2,#3,#4\right)}


%%% Particle things %%%
\newcommand{\pss}[1]{^{(#1)}}

%%% Densities %%%
\newcommand{\den}{p}
\newcommand{\pden}{\hat{p}}
%\newcommand{\priorden}{\varpi}
\newcommand{\transden}{f}
\newcommand{\obsden}{g}
%\newcommand{\loglhood}{\mathcal{L}}

%%% Particle filter things %%%
%\newcommand{\impden}[1]{q_{#1}}
\newcommand{\partden}[1]{\eta_{#1}}
\newcommand{\pw}[1]{w_{#1}}
\newcommand{\npw}[1]{\bar{w}_{#1}}
\newcommand{\predpw}[1]{\hat{w}_{#1}}
\newcommand{\naw}[1]{\bar{v}_{#1}}
\newcommand{\respw}[1]{\tilde{w}_{#1}}
\newcommand{\anc}[2]{a_{#1}^{(#2)}}
\newcommand{\replic}[2]{R_{#1}^{(#2)}}
\newcommand{\ess}[1]{\mathbf{N}_{\text{EFF},#1}}
\newcommand{\lsrm}[1]{\tilde{x}_{#1}}

%%% MCMC things %%%
\newcommand{\mhkernel}{K}
\newcommand{\mhaccept}{\alpha}
\newcommand{\nummhsteps}{M}

%%% Monte Carlo things %%%
\newcommand{\numpart}{\mathbf{N}}
\newcommand{\testfunc}{\zeta}
\newcommand{\unnormden}{\gamma}

%%% State space models %%%
\newcommand{\transfun}{\phi}
\newcommand{\obsfun}{\psi}
\newcommand{\transinnov}[1]{u_{#1}}
\newcommand{\obsinnov}[1]{v_{#1}}

%%% Linear Gaussian model %%%
\newcommand{\lgmtm}{F}
\newcommand{\lgmtnm}{G}
\newcommand{\lgmtv}{Q}
\newcommand{\lgmom}{H}
\newcommand{\lgmov}{R}


%%% MACROS FOR MATHEMATICAL NOTATION IN COMPOSITE PROPOSAL PAPER %%%

% Functions and operators
\newcommand{\magdet}[1]{\left| #1 \right|}         % Magnitude of the determinant
\newcommand{\normal}[3]{\mathcal{N}\left(#1\left|#2,#3\right.\right)}       % Normal density
\newcommand{\studentt}[4]{\mathcal{ST}\left(#1|#2,#3,#4\right)} % Student-t density
% \newcommand{\npd}[4]{\left.\frac{\partial^{#1} #2}{\partial #3^{#1}}\right|_{#4}}
\newcommand{\pdv}[2]{\frac{\partial #1}{\partial #2}}
\newcommand{\ppdv}[2]{\frac{\partial^2 #1}{\partial #2^2}}
\newcommand{\mpdv}[3]{\frac{\partial^2 #1}{\partial #2 \partial #3}}
\newcommand{\npdv}[3]{\frac{\partial^{#1} #2}{\partial #3^{#1}}}
\newcommand{\nmpdv}[5]{\frac{\partial^{#1} #3}{\partial^{#2} #4 \partial #5}}

% Basics
% \newcommand{\rt}{t}                             % Real time
\newcommand{\pt}{\lambda}                       % Pseudo-time
\newcommand{\dpt}{\delta\lambda}                % A little bit of pseudo-time
% \newcommand{\ls}[1]{x_{#1}}                     % Latent state
\newcommand{\dls}{\delta x}                     % A little bit of latent state
% \newcommand{\ob}[1]{y_{#1}}                     % Observation
\newcommand{\mix}[1]{\xi_{#1}}                  % Mixing auxiliary variable
\newcommand{\els}[1]{u_{#1}}                    % Extra latent state

% % Particle shizzle
% \newcommand{\pss}[2][]{^{(#2)#1}}               % Particle superscript
% \newcommand{\pw}[1]{w_{#1}}                     % Particle weight
% \newcommand{\predpw}[1]{\hat{w}_{#1}}           % Predictive particle weight
% \newcommand{\npw}[1]{\bar{w}_{#1}}              % Normalised particle weight
% \newcommand{\naw}[1]{\bar{v}_{#1}}              % Normalised auxiliary weight
% \newcommand{\anc}[1]{a_{#1}}                    % Particle ancestor

% Densities
% \newcommand{\transden}{f}                       % Transition density
% \newcommand{\obsden}{g}                         % Observation density
% \newcommand{\impden}{q}                         % Importance density
% \newcommand{\partden}{\eta}                     % Unweighted particle distribution
% \newcommand{\artden}{\rho}                      % Artificial conditional density
\newcommand{\oiden}[1]{\pi_{#1}}                % Optimal importance density
% % \newcommand{\logoiden}[1]{\Xi_{#1}}             % Log of the optimal importance density
% \newcommand{\logoidenapprox}[1]{\hat{\Xi}_{#1}} % Approximation of the log of the optimal importance density
\newcommand{\approxoiden}[2]{\hat{\pi}_{#1|#2}} % Approximation of the optimal importance density
\newcommand{\ctapproxoiden}[1]{\hat{\pi}_{#1}}
\newcommand{\augfiltden}[1]{\tilde{\pi}_{#1}}   % Augmented filtering density
\newcommand{\oinorm}[1]{K_{#1}}                 % Normalising constant for the optimal importance density
\newcommand{\augfiltnorm}[1]{\tilde{K}_{#1}}    % Normalising constant for the augmented filtering density

% Numbers
% \newcommand{\numpart}{N_F}                      % Number of filter particles
% \newcommand{\ess}[1]{N_{E,#1}}                  % Effective sample size

% Models
% \newcommand{\transfun}{\phi}                    % Transition function
% \newcommand{\obsfun}{h}                         % Observation function
% \newcommand{\transcov}{Q}                       % Transition covariance
% \newcommand{\obscov}{R}                         % Observation covariance
% \newcommand{\transmat}{F}                       % Linear transition matrix
% \newcommand{\obsmat}{H}                         % Linear observation matrix
% \newcommand{\transmean}{m}                      % Mean of the transition density (i.e. f(x_{t-1}))
\newcommand{\dof}{\nu}                          % Degrees of freedom of something student-t-ish

% Linear Gaussian things
% \newcommand{\lgoimean}[1]{\mu_{#1}}             % Linear Gaussian optimal importance density mean
% \newcommand{\lgoicov}[1]{\Sigma_{#1}}           % Linear Gaussian optimal importance density covariance
% \newcommand{\stdnorm}[1]{z_{#1}}                % Standard normal R.V.
% \newcommand{\obmn}[1]{\nu_{#1}}
% \newcommand{\obvr}[1]{S_{#1}}
% \newcommand{\obcvr}[1]{C_{#1}}
\newcommand{\obmnapprox}[1]{\hat{\mu}_{#1}}
\newcommand{\obvrapprox}[1]{\hat{\Sigma}_{#1}}
\newcommand{\obcvrapprox}[1]{\hat{C}_{#1}}

% Gaussian Transformation
\newcommand{\lgdecayfunc}{a}                    % Linear Gaussian decay function for OID transformation
\newcommand{\lgexpsf}{\gamma}                   % Linear Gaussian exponential scale factor for OID transformation
\newcommand{\lgupdmeanmat}[1]{\Gamma_{#1}}      % Mean mapping matrix for the OID transformation
\newcommand{\lgupdcov}[1]{\Omega_{#1}}          % Covariance matrix for the OID transformation
\newcommand{\lginfbm}[1]{\epsilon_{#1}}         % Brownian motion for the infinitesimal form of the OID transformation

% Linear Gaussian approximations
%\newcommand{\lgoimeanapprox}[2]{\hat{\mu}_{#1}(#2)}     % Mean of the Gaussian approximation to the OID at time #1 and state #2
%\newcommand{\lgoicovapprox}[2]{\hat{\Sigma}_{#1}(#2)}   % Covariance of the Gaussian approximation to the OID at time #1 and state #2
\newcommand{\lgoimnapprox}[2]{\hat{m}_{#1|#2}}      % Mean of the Gaussian approximation to the OID at time #1 and state #2
\newcommand{\lgoivrapprox}[2]{\hat{P}_{#1|#2}}    % Covariance of the Gaussian approximation to the OID at time #1 and state #2
\newcommand{\ctlgoimnapprox}[1]{\hat{m}_{#1}}
\newcommand{\ctlgoivrapprox}[1]{\hat{P}_{#1}}
%\newcommand{\lgoimeanapprox}[2][]{\ifstrempty{#1}{\hat{\mu}_{#2}}{\hat{\mu}_{#1|#2}}}
%\newcommand{\lgoicovapprox}[2][]{\ifstrempty{#1}{\hat{\Sigma}_{#2}}{\hat{\Sigma}_{#1|#2}}}

\newcommand{\transmeanapprox}[1]{\hat{\transfun}_{#1}} % Approximate transition mean
\newcommand{\lgmtvapprox}[1]{\hat{\lgmtv}_{#1}}   % Approximate transition covariance
\newcommand{\lgmovapprox}[1]{\hat{\lgmov}_{#1}}       % Approximate observation covariance
\newcommand{\lsfixed}{\ls{}^*}                          % Latent state around which we linearise
\newcommand{\logtrans}{M}                               % Log of the transition density
\newcommand{\logobs}{L}                                 % Log of the observation density

% State evolution SDE
\newcommand{\oudrift}[1]{A_{#1}}                % General O-U process drift term
\newcommand{\oudiffuse}[1]{B_{#1}}              % General O-U process diffusion term
\newcommand{\lserror}[2]{e_{#1|#2}}             % State error due to finite sampling



\newcommand{\flowtd}{\alpha}                    % Flow dervation transition density
\newcommand{\flowod}{\mathcal{L}}               % Flow derivation observation density
%\newcommand{\flowdriftapprox}[2]{\hat{\zeta}_{#1|#2}}      % Approximate (actual) particle flow drift
%\newcommand{\flowdiffuseapprox}[2]{\hat{\eta}_{#1|#2}}     % Approximate (actual) particle flow diffusion

% Simulation models - tracking
\newcommand{\pos}[1]{p_{#1}}           % Position
\newcommand{\vel}[1]{v_{#1}}           % Velocity
\newcommand{\bng}[1]{b_{#1}}               % Bearing
\newcommand{\ele}[1]{\eta_{#1}}                 % Elevation
\newcommand{\rng}[1]{r_{#1}}                    % Range
\newcommand{\hei}[1]{h_{#1}}                    % Height
\newcommand{\rngrt}[1]{s_{#1}}                  % Range rate
\newcommand{\terrain}{T}                        % Terrain height

% Simulation models - heartbeats
\newcommand{\amp}[1]{A_{#1}}                    % Amplitude
\newcommand{\wid}[1]{W_{#1}}                    % Width
\newcommand{\del}[1]{\tau_{#1}}                 % Delay
\newcommand{\freq}[1]{\omega_{#1}}              % Width
\newcommand{\pha}[1]{\psi_{#1}}                 % Phase
\newcommand{\bias}[1]{B_{#1}}                   % Bias





%%% Importance Densities %%%
\newcommand{\lgoimn}[1]{m_{#1}}
\newcommand{\lgoivr}[1]{P_{#1}}
\newcommand{\logoiden}[1]{\Xi_{#1}}
\newcommand{\logoidenapprox}[1]{\hat{\Xi}_{#1}}



\newcommand{\stdnorm}[1]{z_{#1}}
\newcommand{\artden}[1]{\rho_{#1}}
\newcommand{\fixed}{^*}


%%% Gaussian filter parameters %%%
\newcommand{\lsmn}[1]{m_{#1}}
\newcommand{\lsvr}[1]{P_{#1}}
\newcommand{\lspredmn}[1]{\hat{m}_{#1}}
\newcommand{\lspredvr}[1]{\hat{P}_{#1}}
\newcommand{\obmn}[1]{\mu_{#1}}
\newcommand{\obvr}[1]{\Sigma_{#1}}
\newcommand{\obcvr}[1]{C_{#1}}




%%% Environments %%%
\newenvironment{meta}[0]{\color{red} \em}{}



%%% Titles and stuff %%%
\title{Thoughts on Particle Flow}
\author{Pete Bunch}
\date{May 2014}



%%% DOCUMENT %%%

\begin{document}

\maketitle

\section{Introduction}

Consider the task of sampling from a Bayesian posterior distribution over a hidden state variable $\ls{} \in \lsspace = \real^\lsdim$,
%
\begin{IEEEeqnarray}{rCl}
 \postden(\ls{}) & = & \frac{ \priorden(\ls{}) \lhood(\ls{}) }{ \nconst{} } \\
 \nconst{} & = & \int \priorden(\ls{}) \lhood(\ls{}) d\ls{}      .
\end{IEEEeqnarray}
%
in which $\priorden$ and $\postden$ are the prior and posterior densities respectively, which are assumed to exist, $\lhood$ is the likelihood and $\nconst{}$ is a normalising constant. The posterior density is often only available up to a constant of proportionality since this constant cannot be evaluated.

The general principle behind particle flows is to begin with samples from the prior, then to move these according to some dynamics over an interval of pseudo-time such that the final values are distributed according to the posterior. One suitable way to achieve this is to define the following geometric density sequence over the pseudo-time interval $\pt \in \left[0,1\right]$,
%
\begin{IEEEeqnarray}{rCl}
 \seqden{\pt}(\ls{\pt}) & = & \frac{ \priorden(\ls{\pt}) \lhood(\ls{\pt})^{\pt} }{ \nconst{\pt} } \label{eq:density_sequence} \\
 \nconst{\pt}           & = & \int \priorden(\ls{}) \lhood(\ls{})^{\pt} d\ls{}      .
\end{IEEEeqnarray}
%
Since $\seqden{0} = \priorden$, initial particles may be sampled from the prior. These are then moved according to a stochastic differential equation(SDE) such that at every instant in pseudo-time each one is distributed according to the appropriate density in the sequence \eqref{eq:density_sequence},
%
\begin{IEEEeqnarray}{rCl}
 d\ls{\pt} & = & \flowdrift{\pt}(\ls{\pt}) d\pt + \flowdiffuse{\pt} d\flowbm{\pt} \label{eq:state_sde}     .
\end{IEEEeqnarray}

At the end, since $\seqden{1} = \postden$, the final particles are distributed according to the posterior. These can be used to construct an unbiased, asymptotically consistent estimator of any posterior expectation. Hooray.

The challenge in applying such a particle flow sampler comes in finding suitable dynamics with which to move the particles such that the correct density is maintained throughout. In general, this cannot be achieved analytically, and approximations are called for. While these may be effective, they result in the loss of asymptotic consistency, and the introduction of bias which is not easily quantified.

It may be shown that the SDE drift and diffusion need to satisfy the following governing equation,
%
\begin{IEEEeqnarray}{rCl}
 \loglhood(\ls{\pt}) - \expect{\seqden{\pt}}\left[ \loglhood \right] & = & -\trace\left[ \pdv{\flowdrift{\pt}}{\ls{\pt}} \right] - \flowdrift{\pt}(\ls{\pt})^T \pdv{\logseqden{\pt}}{\ls{\pt}} \nonumber \\
 & & \qquad + \: \trace\left[ \flowcov{\pt} \ppdv{\logseqden{\pt}}{\ls{\pt}} \right] + \pdv{\logseqden{\pt}}{\ls{\pt}}^T \flowcov{\pt} \pdv{\logseqden{\pt}}{\ls{\pt}} \label{eq:optimal_flow_pde}        ,
\end{IEEEeqnarray}
%
in which
%
\begin{IEEEeqnarray}{rCl}
 \logseqden{\pt}(\ls{\pt}) & = & \log(\seqden{\pt}(\ls{\pt})) \nonumber \\
 \loglhood(\ls{\pt})  & = & \log(\lhood(\ls{\pt}))  \nonumber \\
 \flowcov{\pt}             & = & \half \flowdiffuse{\pt} \flowdiffuse{\pt}^T \nonumber \\
 \expect{\seqden{\pt}}\left[ \loglhood \right] & = & \int \seqden{\pt}(\ls{}) \loglhood(\ls{}) d\ls{} \label{eq:optimal_flow_pde_terms}      .
\end{IEEEeqnarray}

\emph{All} we need to do is choose $\flowdrift{\pt}$ and $\flowdiffuse{\pt}$ which satisfy \eqref{eq:optimal_flow_pde}, then analytically integrate \label{eq:state_sde} to find the final particle locations.


\section{Thoughts and Observations}

Fred Daum has a number of particle flows which he uses to implement particle filters. In deriving this assortment, he routinely makes three separate approximations:
%
\begin{enumerate}
  \item He chooses $\flowdrift{\pt}$ and $\flowdiffuse{\pt}$ which do not satisfy \eqref{eq:optimal_flow_pde}. He does this because this partial differential equation (PDE) is actually really hard to solve, even though it is highly underdetermined. \label{item:approx-flow}
  \item He integrates \eqref{eq:state_sde} numerically. \label{item:approx-integration}
  \item Since he applies the particle flow directly to the filtering density, the prior $\priorden(\ls{})$ is the predictive density which is not known. So he approximates this, along with its gradient. \label{item:approx-prior}
\end{enumerate}

Some observations:
%
\begin{itemize}
  \item The desirable property of his filters, that they maintain equal weights throughout and hence evade particle degeneracy, actually arises from approximation~\ref{item:approx-prior}, and not directly from the use of particle flow. Is there a non-flow way to use this approximation?
  \item We can always correct for the discrepancy between the actual and ideal particle densities introduced by approximation~\ref{item:approx-flow} using an importance weight. All we need is a Jacobian of the state update, which comes from the integrated SDE.
  \item Of course, in reality the only useful flow which can be integrated analytically is the Gaussian case. The rest of the time we will need numerical integration. We can still use importance sampling to correct for the errors that this introduces, as long as we can calculate appropriate Jacobians for the state changes. This is \emph{not} the case if we use numerical approximations of $\flowdrift{\pt}$ and $\flowdiffuse{\pt}$.
\end{itemize}

Some lines for future work:
%
\begin{itemize}
  \item Try out existing flows other than the Gaussian case.
  \item Think about finding an exact solution to \eqref{eq:optimal_flow_pde} and approximating the required integrals numerically, rather than just ignoring or making assumptions about the various terms.
  \item Deduce Jacobians for numerical integration schemes more complex than Euler.
  \item Try out other approximations of the prior and likelihood which allow a more robust application of the Gaussian flow.
  \item Think about the effect of the prior approximation. Is it somehow equivalent to resampling? Can it be used in other ways?
\end{itemize}



\section{Jacobians for Numerical Integration}

If we integrate our SDE \eqref{eq:state_sde} numerically, then we need an appropriate Jacobian for the transformation in order to calculate an appropriate importance weight. The simplest possibility is to set $\flowcov{\pt}=0$ so we just have an ODE and then use Euler integration,
%
\begin{IEEEeqnarray}{rCl}
 \ls{k+1} & = & \ls{k} + \flowdrift{\pt_k}(\ls{k}) \dpt \nonumber      .
\end{IEEEeqnarray}
%
Here the Jacobian is easy,
%
\begin{IEEEeqnarray}{rCl}
 \determ{\frac{\ls{k+1}}{\ls{k}}} & = & \determ{ I + \pdv{\flowdrift{\pt_k}}{\ls{\pt_k}} \dpt } \nonumber \\
 & \approx & 1 + \dpt \trace\left[\pdv{\flowdrift{\pt_k}}{\ls{\pt_k}}\right]
\end{IEEEeqnarray}
%
As long as this can be calculated, then we're in business.

A similar scheme is possible for all explicit Runge-Kutta methods. Just write the update as a single equation and then differentiate. An $n$th order method will require $n$ Jacobians to be calculated.

When $\flowcov{\pt}>0$, the simplest choice is to use the Euler-Maruyama method. This is equivalent to sampling a Gaussian random variable, and so the same extended distribution trick may be used as in the Gaussian flow case. For higher order Runge-Kutta methods, it becomes necessary to extend the distribution over all the intermediate integration points. This will give us a proper weight, but I suspect it is not likely to work well.



\section{The 1D Case}

Suppose we work in one dimension and set $\flowcov{\pt} = 0$. Then we can solve \eqref{eq:optimal_flow_pde}. It becomes,
%
\begin{IEEEeqnarray}{rCl}
 \loglhood(\ls{\pt}) - \expect{\seqden{\pt}}\left[ \loglhood \right] & = & -\pdv{\flowdrift{\pt}}{\ls{\pt}} - \flowdrift{\pt}(\ls{\pt}) \pdv{\logseqden{\pt}}{\ls{\pt}} \label{eq:optimal_flow_pde_1D}      .
\end{IEEEeqnarray}
%
Use the integrating factor method. Multiplying the right hand side by $\seqden{\pt}$,
%
\begin{IEEEeqnarray}{rCl}
 \seqden{\pt}(\ls{\pt}) \left[ - \pdv{\flowdrift{\pt}}{\ls{\pt}} - \flowdrift{\pt}(\ls{\pt}) \pdv{\logseqden{\pt}}{\ls{\pt}} \right] & = & - \left[ \seqden{\pt}(\ls{\pt}) \pdv{\flowdrift{\pt}}{\ls{\pt}} - \flowdrift{\pt}(\ls{\pt}) \pdv{\seqden{\pt}}{\ls{\pt}} \right] \nonumber \\
 & = & - \pdv{}{\ls{\pt}} \left[ \seqden{\pt}(\ls{\pt}) \flowdrift{\pt}(\ls{\pt}) \right] \nonumber     ,
\end{IEEEeqnarray}
%
and so integrating and with a little algebra,
%
\begin{IEEEeqnarray}{rCl}
 - \seqden{\pt}(\ls{\pt}) \flowdrift{\pt}(\ls{\pt}) & = & \int_{-\infty}^{\ls{\pt}} \seqden{\pt}(\ls{\pt}) \left\{\loglhood(z) - \expect{\seqden{\pt}}\left[ \loglhood \right] \right\} dz \nonumber \\
 & = & \int_{-\infty}^{\ls{\pt}} \seqden{\pt}(z) \loglhood(z) dz \int_{\ls{\pt}}^{\infty} \seqden{\pt}(z) dz + \int_{\ls{\pt}}^{\infty} \seqden{\pt}(z) \loglhood(z) dz \int_{-\infty}^{\ls{\pt}}  \seqden{\pt}(z) dz \nonumber      ,
\end{IEEEeqnarray}
%
and so,
%
\begin{IEEEeqnarray}{rCl}
 \flowdrift{\pt}(\ls{\pt}) & = & - \frac{ \int_{-\infty}^{\ls{\pt}} \seqden{\pt}(z) \loglhood(z) dz \int_{\ls{\pt}}^{\infty} \seqden{\pt}(z) dz + \int_{\ls{\pt}}^{\infty} \seqden{\pt}(z) \loglhood(z) dz \int_{-\infty}^{\ls{\pt}}  \seqden{\pt}(z) dz }{ \seqden{\pt}(\ls{\pt}) } \nonumber      .
\end{IEEEeqnarray}
%
We could estimate this using Monte Carlo using the current set of particles. Furthermore, since $\pdv{\flowdrift{\pt}}{\ls{\pt}}$ is given by \eqref{eq:optimal_flow_pde_1D}, we can clearly calculate the appropriate Jacobian for simpler numerical integration schemes (e.g. Euler).

The problems is that there is no obvious extension to higher dimensions.






\end{document} 