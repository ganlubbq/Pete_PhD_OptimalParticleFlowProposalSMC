\documentclass{statsoc}

%%% Packages %%%

% Graphics
\usepackage{graphicx}
\usepackage[caption=false,font=footnotesize]{subfig}

% Formatting
\usepackage{color}
\usepackage{amsmath}
\usepackage{amsfonts}
\usepackage{bbm}
\usepackage[T1]{fontenc}

% Environments
\usepackage{IEEEtrantools}
\usepackage{algorithm}
\usepackage{algorithmic}

% References
\usepackage{natbib}



%%% Macros %%%
%%% MACROS FOR MATHEMATICAL NOTATION IN COMPOSITE PROPOSAL PAPER %%%

% Functions and operators
\newcommand{\half}{\frac{1}{2}}                                 % Half
\newcommand{\expect}[1]{\mathbb{E}_{#1}}                        % Expectation
\newcommand{\variance}[1]{\mathbb{V}_{#1}}                      % Variance
\DeclareMathOperator{\trace}{Tr}                                % Trace
\newcommand{\magdet}[1]{\left| #1 \right|}         % Magnitude of the determinant
\newcommand{\indic}[1]{\mathbbm{1}_{#1}}                        % Indicator function
\newcommand{\normal}[3]{\mathcal{N}\left(#1\left|#2,#3\right.\right)}       % Normal density
\newcommand{\gammaden}[3]{\mathcal{\Gamma}\left(#1|#2,#3\right)}% Gamma density
\newcommand{\studentt}[4]{\mathcal{ST}\left(#1|#2,#3,#4\right)} % Student-t density
\newcommand{\bigo}[1]{\mathcal{O}\left({#1}\right)}             % Big O
\newcommand{\mhaccept}{\alpha}                                  % Metropolis-Hastings acceptance probability

% Basics
\newcommand{\rt}{t}                             % Real time
\newcommand{\pt}{\lambda}                       % Pseudo-time
\newcommand{\dpt}{\delta\lambda}                % A little bit of pseudo-time
\newcommand{\ls}[1]{x_{#1}}                     % Latent state
\newcommand{\ob}[1]{y_{#1}}                     % Observation
\newcommand{\mix}[1]{\xi_{#1}}                  % Mixing auxiliary variable
\newcommand{\els}[1]{u_{#1}}                    % Extra latent state

% Particle shizzle
\newcommand{\pss}[2][]{^{(#2)#1}}               % Particle superscript
\newcommand{\pw}[1]{w_{#1}}                     % Particle weight
\newcommand{\predpw}[1]{\hat{w}_{#1}}           % Predictive particle weight
\newcommand{\npw}[1]{\bar{w}_{#1}}              % Normalised particle weight
\newcommand{\naw}[1]{\bar{v}_{#1}}              % Normalised auxiliary weight
\newcommand{\anc}[1]{a_{#1}}                    % Particle ancestor

% Densities
\newcommand{\transden}{f}                       % Transition density
\newcommand{\obsden}{g}                         % Observation density
\newcommand{\impden}{q}                         % Importance density
\newcommand{\partden}{\eta}                     % Unweighted particle distribution
\newcommand{\artden}{\rho}                      % Artificial conditional density
\newcommand{\oiden}[1]{\pi_{#1}}                % Optimal importance density
\newcommand{\approxoiden}[2]{\hat{\pi}_{#1|#2}} % Approximation of the optimal importance density
\newcommand{\augfiltden}[1]{\tilde{\pi}_{#1}}   % Augmented filtering density
\newcommand{\oinorm}[1]{K_{#1}}                 % Normalising constant for the optimal importance density
\newcommand{\augfiltnorm}[1]{\tilde{K}_{#1}}    % Normalising constant for the augmented filtering density

% Numbers
\newcommand{\numpart}{N_F}                      % Number of filter particles
\newcommand{\ess}[1]{N_{E,#1}}                  % Effective sample size

% Models
\newcommand{\transfun}{\phi}                    % Transition function
\newcommand{\obsfun}{h}                         % Observation function
\newcommand{\transcov}{Q}                       % Transition covariance
\newcommand{\obscov}{R}                         % Observation covariance
\newcommand{\transmat}{F}                       % Linear transition matrix
\newcommand{\obsmat}{H}                         % Linear observation matrix
\newcommand{\transmean}{m}                      % Mean of the transition density (i.e. f(x_{t-1}))
\newcommand{\dof}{\nu}                          % Degrees of freedom of something student-t-ish

% Linear Gaussian things
\newcommand{\lgoimean}[1]{\mu_{#1}}             % Linear Gaussian optimal importance density mean
\newcommand{\lgoicov}[1]{\Sigma_{#1}}           % Linear Gaussian optimal importance density covariance
\newcommand{\stdnorm}[1]{z_{#1}}                % Standard normal R.V.

% Gaussian Transformation
\newcommand{\lgdecayfunc}{a}                    % Linear Gaussian decay function for OID transformation
\newcommand{\lgexpsf}{\gamma}                   % Linear Gaussian exponential scale factor for OID transformation
\newcommand{\lgupdmeanmat}[1]{\Gamma_{#1}}      % Mean mapping matrix for the OID transformation
\newcommand{\lgupdcov}[1]{\Omega_{#1}}          % Covariance matrix for the OID transformation
\newcommand{\lginfbm}[1]{\epsilon_{#1}}         % Brownian motion for the infinitesimal form of the OID transformation

% Linear Gaussian approximations
%\newcommand{\lgoimeanapprox}[2]{\hat{\mu}_{#1}(#2)}     % Mean of the Gaussian approximation to the OID at time #1 and state #2
%\newcommand{\lgoicovapprox}[2]{\hat{\Sigma}_{#1}(#2)}   % Covariance of the Gaussian approximation to the OID at time #1 and state #2
\newcommand{\lgoimeanapprox}[2]{\hat{\mu}_{#1|#2}}      % Mean of the Gaussian approximation to the OID at time #1 and state #2
\newcommand{\lgoicovapprox}[2]{\hat{\Sigma}_{#1|#2}}    % Covariance of the Gaussian approximation to the OID at time #1 and state #2
\newcommand{\obsmatapprox}[1]{\hat{\obsmat}_{#1}}       % Linear observation matrix formed by differentiation of the observation function
\newcommand{\transmeanapprox}[1]{\hat{\transmean}_{#1}} % Approximate transition mean
\newcommand{\transcovapprox}[1]{\hat{\transcov}_{#1}}   % Approximate transition covariance
\newcommand{\obapprox}[1]{\hat{y}_{#1}}                 % Approximate observation mean
\newcommand{\obscovapprox}[1]{\hat{\obscov}_{#1}}       % Approximate observation covariance
\newcommand{\lsfixed}{\ls{}^*}                          % Latent state around which we linearise
\newcommand{\logtrans}{L}                               % Log of the transition density
\newcommand{\logobs}{M}                                 % Log of the observation density

% State evolution SDE
\newcommand{\oudrift}[1]{A_{#1}}                % General O-U process drift term
\newcommand{\oudiffuse}[1]{B_{#1}}              % General O-U process diffusion term
\newcommand{\sdedrift}[1]{\zeta_{#1}}           % Drift
\newcommand{\lserror}[2]{e_{#1|#2}}             % State error due to finite sampling

% Particle flow
\newcommand{\flowbm}[1]{\epsilon_{#1}}          % Particle flow Brownian motion
\newcommand{\flowdrift}[1]{\zeta_{#1}}          % Particle flow drift
\newcommand{\flowdiffuse}[1]{\eta_{#1}}         % Particle flow diffusion
\newcommand{\flowcov}[1]{D_{#1}}                % Particle flow "Covariance" matrix
\newcommand{\flowtd}{\alpha}                    % Flow dervation transition density
\newcommand{\flowod}{\beta}                     % Flow derivation observation density

% Simulation models - tracking
\newcommand{\pos}[1]{p_{#1}}           % Position
\newcommand{\vel}[1]{v_{#1}}           % Velocity
\newcommand{\bng}[1]{\theta_{#1}}               % Bearing
\newcommand{\rng}[1]{r_{#1}}                    % Range
\newcommand{\hei}[1]{h_{#1}}                    % Height
\newcommand{\rngrt}[1]{s_{#1}}                  % Range rate
\newcommand{\terrain}{T}                        % Terrain height

% Simulation models - heartbeats
\newcommand{\amp}[1]{A_{#1}}                    % Amplitude
\newcommand{\wid}[1]{W_{#1}}                    % Width
\newcommand{\del}[1]{\tau_{#1}}                 % Delay
\newcommand{\freq}[1]{\omega_{#1}}              % Width
\newcommand{\pha}[1]{\psi_{#1}}                 % Phase
\newcommand{\bias}[1]{B_{#1}}                   % Bias





%%% Environments %%%
\newenvironment{meta}[0]{\color{red} \em}{}
\newtheorem{lemma}{Lemma}



%%% Titles and stuff %%%
\title[Composite Proposal Particle Filter]{The Composite Proposal Particle Filter: Better Approximations to the Optimal Importance Density}
\author[Bunch {\it et al.}]{Pete Bunch}
\address{Cambridge University Engineering Department,Cambridge,UK.}
\email{pb404@cam.ac.uk}



%%% DOCUMENT %%%

\begin{document}



\begin{abstract}
Abstract here.
\end{abstract}



\keywords{particle filter, sequential Monte Carlo, optimal importance distribution}



\section{Introduction}

A particle filter is an algorithm used for sequential estimation of a filtering distribution for a state-space model. For a comprehensive introduction, see for example \citep{Cappe2007,Doucet2009}. In this paper we consider the use of particle filters for inference with a standard discrete-time hidden Markov model (HMM).

The particle filter advances a set of samples through time, drawn approximately from the filtering distribution. This is achieved by sampling at each time step from an importance distribution and then weighting the particles to account for the discrepancy between target and importance distributions. Particle filters have attractive asymptotic properties: as the number of particles is increased, certain estimates are guaranteed to converge to their true values. % Resampling steps are used to control the variance of the importance weights, and prevent a single particle from dominating the set.

One of the principal difficulties when designing a particle filter is the selection of the importance distribution. The simplest choice is often to sample from the transition model, resulting in the ``bootstrap filter'' of \citep{Gordon1993}. In many cases, such bootstrap proposals result in poor filter performance due to a mismatch in the areas of high probability between the transition and observation distributions.

Amongst others, \citet{Doucet2000a} demonstrated that the ideal choice of importance distribution for each particle is the conditional posterior given both the previous state and the new observation, dubbed the ``optimal importance distribution'' (OID). In all but a few cases, this cannot be calculated analytically. When the state variables are continuous, a popular solution is to use an extended (EKF) or unscented (UKF) Kalman filter to select a Gaussian importance density for each particle. However, such schemes can fail when the model is highly nonlinear or non-Gaussian, as the approximation is poor.

The effect of using a bad importance distribution (i.e. one which is not ``close'' to the OID) is that the variance of the importance weights is high, resulting in a degeneracy of the filter. In the worst cases, there may be no particles at all proposed in regions of high posterior probability, causing the filter to fail entirely. This problem is especially pronounced when the dimensionality of the state space is high --- there is simply more space for the particles to cover.

Numerous additions and modifications to the basic particle filter have been proposed. For example, with some models it may be possible to marginalise a subset of the state variables --- a process known as ``Rao-Blackwellisation'' --- and hence to reduce the dimensionality of the samples in the particle filter \citep{Casella1996,Doucet2000}.

A more generally applicable enhancement is to insert Markov chain Monte Carlo (MCMC) steps to rejuvenate a degenerate set of particles, a method named ``resample-move'' by \citet{Gilks2001}. However, for this to work effectively and efficiently, it is necessary that the preceding importance sampling stage manages to place at least $1$ particle in each interesting region of the state space (i.e. where the posterior probability is high). In addition, an MCMC stage introduces new algorithm parameters which need to be tuned for effective operation: e.g. the number of MCMC steps per particle, and the proposal distribution.

Another way in which degeneracy may be mitigated is by introducing the effect of each observation gradually, so that particles may be progressively drawn towards peaks in the likelihood. This can be achieved by using a discrete set of bridging distributions which transition smoothly between the prior and posterior. Each one is targeted in turn using importance sampling and particle diversity is maintained using Metropolis-Hastings moves. Such ``annealing'' schemes have been suggested by, amongst others, \citet{Neal2001} (using MCMC) and \citet{DelMoral2006} (using Sequential Monte Carlo (SMC) samplers) for static inference problems, and by \citet{Godsill2001b,Gall2007,Deutscher2000} for particle filters.

It is possible to take the idea of bridging distributions to a limit and define a continuous sequence of distributions between the prior and the posterior. This device was used by \citet{Gelman1998} for the related task of simulating normalising constants, and has been used to design sophisticated assumed density filters \citep{Hanebeck2003a,Hanebeck2012,Hagmar2011}. More recently, particle filters have appeared which exploit the same principle, including the ``particle flow'' methods described in series of papers including \citep{Daum2008,Daum2011d}, and the ``optimal transport'' methods of \cite{Reich2011,Reich2012a}.

\subsection{Composite Proposals}

In this paper, a new method is introduced for sampling from an approximation of the optimal importance density, which we name the composite proposal particle filter (CPPF). The purpose of the CPPF is to provide more efficient proposals for particularly challenging problems, on which standard particle filters fail. In common with progressive filtering and particle flow algorithms, the procedure relies on introducing the observation likelihood gradually. Beginning with a sample from the transition density, a series of local approximations are then used to move the particle to a new state, either deterministically or stochastically. This method is shown to yield significant improvements in effective sample size for challenging nonlinear models.

A modification of the algorithm is also presented which introduces intermediate resampling steps during the smooth update. This helps further to limit degeneracy of the particle weights, at the cost of introducing dependency between the particles. This latter algorithm resembles (and yet is significantly different from) an annealed particle filter \citep{Gall2007,Deutscher2000}, but uses gradient information from the model densities to lead particles towards peaks in the OID. In this sense, there are also connections with adaptive MCMC methods, such as those of \citep{Girolami2011}.



\section{Particle Filtering}

\subsection{Basics}\label{sec:pf_basics}

We consider a standard discrete-time HMM in which the transition, observation and prior models have closed-form densities,
%
\begin{IEEEeqnarray}{rCl}
 \ls{\rt} & \sim & p(\ls{\rt} | \ls{\rt-1}) \label{eq:td} \\
 \ob{\rt} & \sim & p(\ob{\rt} | \ls{\rt})   \label{eq:od} \\
 \ls{1} & \sim & p(\ls{1})                  \label{eq:pd}      ,
\end{IEEEeqnarray}
%
where the random variable $\ls{\rt}$ is the hidden state of a system at time $\rt$, and $\ob{\rt}$ is an incomplete, noisy observation. We assume here that the transition, observation and prior densities may be evaluated and that the prior and transition densities may be sampled. A particle filter is used to estimate recursively distributions over the path of the state variables, $\ls{1:\rt}=\{\ls{1}, \dots, \ls{\rt}\}$. Densities are approximated by a sum of weighted probability masses located at a discrete set of states,
%
\begin{IEEEeqnarray}{rCl}
 p(\ls{1:\rt} | \ob{1:k}) & = & \sum_i \npw{\rt}\pss{i} \delta_{\ls{1:\rt}\pss{i}}(\ls{1:\rt})     ,
\end{IEEEeqnarray}
%
where $\delta_{\ls{1:\rt}\pss{i}}(\ls{1:\rt})$ denotes a unit probability mass at the point $\ls{1:\rt}\pss{i}$.

The particle filter recursion may be separated into two stages --- prediction and update --- which produce approximations to the predictive, $p(\ls{1:\rt} | \ob{1:\rt-1})$, and filtering, $p(\ls{1:\rt} | \ob{1:\rt})$, densities respectively. (Note, these terms more conventionally refer to the density of the latest state only, rather than the entire path.)

At time $\rt$, the prediction stage begins with selection (resampling) of a parent from amongst the $\rt-1$ particles; an index, $\anc{j}$, is chosen with probability (auxiliary weight) $\naw{t-1}\pss{j}$. This is equivalent to sampling from a density $\impden(\ls{1:\rt-1} | \ob{1:\rt-1})$, where,
%
\begin{IEEEeqnarray}{rCl}
 \impden(\ls{1:\rt-1}\pss{i} | \ob{1:\rt-1}) & \propto & \frac{ \naw{\rt-1}\pss{i} }{ \npw{\rt-1}\pss{i} } p(\ls{1:\rt-1}\pss{i} | \ob{1:\rt-1}) \nonumber      .
\end{IEEEeqnarray}

Next, a new state $\ls{\rt}\pss{j}$ is sampled from an importance density, $\impden(\ls{\rt} | \ls{\rt-1}\pss{\anc{j}}, \ob{\rt})$, and concatenated to the parent path to form the new particle,
%
\begin{IEEEeqnarray}{rCl}
 \ls{1:\rt}\pss{j} \leftarrow \left\{ \ls{1:\rt-1}\pss{\anc{j}},  \ls{\rt}\pss{j} \right\}     .
\end{IEEEeqnarray}
%
Finally, an importance weight is assigned to the particle to account for the discrepancy between importance and target distributions,
%
\begin{IEEEeqnarray}{rCl}
 \predpw{\rt}\pss{j} & = & \frac{ p(\ls{1:\rt}\pss{j} | \ob{1:\rt-1}) }{ \impden(\ls{1:\rt-1}\pss{\anc{j}} | \ob{1:\rt-1}) \impden(\ls{\rt}\pss{j} | \ls{\rt-1}\pss{\anc{j}}, \ob{\rt}) } \nonumber \\
 & \propto & \frac{\npw{\rt-1}\pss{j}}{\naw{t-1}\pss{\anc{j}}} \times \frac{ p(\ls{\rt}\pss{j} | \ls{\rt-1}\pss{\anc{j}}) }{ \impden(\ls{\rt}\pss{j} | \ls{\rt-1}\pss{\anc{j}}, \ob{\rt}) }     .
\end{IEEEeqnarray}

In the update stage, the same set of particles is used to approximate the filtering distribution. These are distributed according to $\impden(\ls{1:\rt} | \ob{1:\rt})$, where,
%
\begin{IEEEeqnarray}{rCl}
 p(\ls{1:\rt}\pss{j} | \ob{1:\rt-1}) & \propto & \predpw{\rt}\pss{j} \impden(\ls{1:\rt}\pss{j} | \ob{1:\rt}) \nonumber      .
\end{IEEEeqnarray}
%
A new importance weight is required to account for the discrepancy,
%
\begin{IEEEeqnarray}{rCl}
 \pw{\rt}\pss{j} & =       & \frac{ p(\ls{1:\rt}\pss{j} | \ob{1:\rt}) }{ \impden(\ls{1:\rt}\pss{j} | \ob{1:\rt}) } \nonumber \\
                 & \propto & \predpw{\rt}\pss{j} \times p(\ob{\rt} | \ls{\rt}\pss{j} )      .
\end{IEEEeqnarray}
%
Finally, the weights are normalised,
%
\begin{IEEEeqnarray}{rCl}
 \npw{\rt} & = & \frac{ \pw{\rt}\pss{j} }{ \sum_i \pw{\rt}\pss{i} }      .
\end{IEEEeqnarray}

If the two steps are considered together, then the combined weight update is,
%
\begin{IEEEeqnarray}{rCl}
 \pw{\rt}\pss{j} & = & \frac{ p(\ls{1:\rt}\pss{j} | \ob{1:\rt}) }{ \impden(\ls{1:\rt-1}\pss{\anc{j}} | \ob{1:\rt-1}) \impden(\ls{\rt}\pss{j} | \ls{\rt-1}\pss{\anc{j}}, \ob{\rt}) } \nonumber \\
 & \propto & \frac{\npw{\rt-1}\pss{j}}{\naw{t-1}\pss{\anc{j}}} \times \frac{ p(\ob{\rt} | \ls{\rt}\pss{j}) p(\ls{\rt}\pss{j} | \ls{\rt-1}\pss{\anc{j}}) }{ \impden(\ls{\rt}\pss{j} | \ls{\rt-1}\pss{\anc{j}}, \ob{\rt}) }     .
\end{IEEEeqnarray}

Particle filters have the highly desirable property of asymptotic consistency; as the number of particles tends to infinite, integrals of bounded functions over the filtering density converge to the true value.

The practical performance of the particle filter is determined by the variance of the weights. If this is high, then only a small proportion of the particles (perhaps only one) will be significant, and only these will be taken forward to the next filtering step. Clearly, a lower number of significant particles leads to a poorer representation of the distribution, resulting in an increased estimator variance and propensity for the filter to diverge completely from the true value. The particle weight variance may be measured using the effective sample size (ESS), defined as,
%
\begin{IEEEeqnarray}{rCl}
 \ess{\rt} & = & \frac{ 1 }{ \sum_i \npw{\rt}\pss[2]{i} }     ,
\end{IEEEeqnarray}
%
Intuitively, this is the number of particles which would be present in an equivalent set comprised of independent, unweighted samples. It takes a value between $1$ (which is bad) and the number of filtering particles, $\numpart$ (which is good).

\subsection{Design Considerations}

The remaining considerations in the design of a particle filter are the choices of the auxiliary weights, $\naw{\rt-1}\pss{j}$, and the importance density, $\impden(\ls{\rt} | \ls{\rt-1},\ob{\rt})$. The options for particle selection are:
\begin{enumerate}
  \item No resampling: Use every particle from the $t-1$ set exactly once. This is achieved by setting $\anc{j}=j$ and corresponds to using $\naw{\rt-1}\pss{j}=\frac{1}{\numpart}$. This uses the least computation and introduces no Monte Carlo error. However, if used exclusively, it will allow the weight variance to accumulate over time and the ESS is guaranteed to fall to $1$.
  \item Standard resampling: Use $\naw{\rt-1}\pss{j}=\npw{\rt-1}\pss{j}$ and sample $\{\anc{i}\}$ using multinomial or systematic resampling \citep{Hol2006}. This prevents the accumulation of weight variance at the cost of some additional Monte Carlo error. It also introduces dependence between the particles.
  \item Auxiliary resampling: Set $\naw{\rt-1}\pss{j}$ so as to minimise the variance of the $\npw{\rt}\pss{j}$ \citep{Pitt1999}. In combination with the optimal importance density, this leads to an equally weighted set of particles. However, this requires the calculation of an integral over the state space which is generally intractable, and approximations of which can lead to worse performance than the simpler options. Therefore, it is not considered here.
\end{enumerate}

Various choices also exist for importance density. The simplest is to use the transition density,
%
\begin{IEEEeqnarray}{rCl}
 \impden(\ls{\rt} | \ls{\rt-1}\pss{\anc{j}}, \ob{\rt}) = p(\ls{\rt} | \ls{\rt-1}\pss{\anc{j}})     .
\end{IEEEeqnarray}
%
This results in the ``bootstrap filter'' of \cite{Gordon1993}. It is only requires that sampling be possible from the transition model, and not that the transition density be calculable. The weight formula simplifies to,
%
\begin{IEEEeqnarray}{rCl}
 \pw{\rt}\pss{j} & \propto & \frac{\npw{\rt-1}\pss{j}}{\naw{\rt-1}\pss{j}} \times p(\ob{\rt} | \ls{\rt}\pss{j}) \label{eq:weight_update_bootstrap}      .
\end{IEEEeqnarray}
%
Furthermore, in combination with standard resampling, the weight associated with the prediction stage is a constant; it is the update stage which is problematic.

Often the bootstrap filter is inefficient, especially when the variance of the transition density is greater than that of the observation density. In this situation, the samples are widely spread over the state space, and only a few fall in the region of high likelihood. This results in a high weight variance, low ESS and poor filter performance.

It was shown in \citep{Doucet2000a}, and references therein, that the weight variance is minimised by using the conditional posterior as the importance distribution,
%
\begin{IEEEeqnarray}{rCl}
 \impden(\ls{\rt} | \ls{\rt-1}\pss{\anc{j}}, \ob{\rt}) & = & p(\ls{\rt} | \ls{\rt-1}\pss{\anc{j}}, \ob{\rt})      ,
\end{IEEEeqnarray}
%
resulting in the following weight formula,
%
\begin{IEEEeqnarray}{rCl}
 \pw{\rt}\pss{j} & \propto & \frac{\npw{\rt-1}\pss{j}}{\naw{\rt-1}\pss{j}} \times p(\ob{\rt} | \ls{\rt-1}\pss{\anc{j}}) \nonumber \\
           & \propto & \frac{\npw{\rt-1}\pss{j}}{\naw{\rt-1}\pss{j}} \times \int p(\ob{\rt} | \ls{\rt}) p(\ls{\rt} | \ls{\rt-1}\pss{\anc{j}}) d\ls{\rt}      .
\end{IEEEeqnarray}
%
This choice is thus known as the ``optimal importance density'' (OID). It may be sampled from, and the weights calculated in closed form, when the observation density is linearly dependent on the state and both transition and observation densities are Gaussian. (The state need not be linearly dependent on the previous state.) However, for most models this density can be neither calculated, nor efficiently sampled from. Thus, it is common to use the same Gaussian approximations to estimate and sample from the OID as were used in the formulation of the EKF and UKF \citep{Doucet2000a,Merwe2000}. These work well when the OID is unimodal, and the observation nonlinearity is weak, but can otherwise perform worse even than the bootstrap filter.



\section{Composite Proposals}

The composite proposal method is a procedure for sampling approximately from the OID by introducing the likelihood progressively and making a series of local Gaussian approximations. In order to achieve this, an auxiliary variable $\pt \in [0,1]$ is used. Intuitively, this is a stretch of ``pseudo-time'' between the predictive and filtering densities, which we link via a continuous sequence of densities,
%
\begin{IEEEeqnarray}{rCl}
 \augfiltden{\rt,\pt}(\ls{1:\rt-1}, \ls{\rt,\pt}) & = & \frac{ p(\ob{\rt} | \ls{\rt,\pt})^{\pt} p(\ls{\rt,\pt} | \ls{\rt-1}) p(\ls{1:\rt-1}|\ob{1:\rt-1}) }{ \augfiltnorm{\pt} } \label{eq:filtering_sequence} \\
 \augfiltnorm{\pt} & = & \int p(\ob{\rt} | \ls{\rt,\pt})^{\pt} p(\ls{\rt,\pt} | \ob{1:\rt-1}) d\ls{\rt,\pt}      ,
\end{IEEEeqnarray}
%
in which $\ls{\rt,\pt}$ is defined as the state at time $\rt$ and pseudo-time $\pt$. This filtering sequence becomes the predictive density when $\pt=0$ and the desired filtering density when $\pt=1$. In the composite proposal method, rather than attempting to sample directly from the OID, a series of incremental updates are applied in order to advance a particle state $\ls{\rt,\pt}\pss{j}$ and its associated weight $\pw{\rt,\pt}\pss{j}$ through pseudo-time, so that it is correctly distributed according to \eqref{eq:filtering_sequence} throughout.

State updates are derived by considering a sequence of optimal importance densities for the particle,
%
\begin{IEEEeqnarray}{rCl}
 \oiden{\rt,\pt}(\ls{\rt,\pt} | \ls{\rt-1}\pss{\anc{j}}) & = & \frac{ p(\ob{\rt} | \ls{\rt,\pt})^{\pt} p(\ls{\rt,\pt} | \ls{\rt-1}\pss{\anc{j}}) }{ \oinorm{\pt}(\ls{\rt-1}\pss{\anc{j}}) } \label{eq:OID_sequence} \\
 \oinorm{\pt}(\ls{\rt-1}\pss{\anc{j}}) & = & \int p(\ob{\rt} | \ls{\rt,\pt})^{\pt} p(\ls{\rt,\pt} | \ls{\rt-1}\pss{\anc{j}}) d\ls{\rt,\pt}      .
\end{IEEEeqnarray}
%
This sequence begins with the transition density at $\pt=0$ and finishes with the OID at $\pt=1$. By considering the evolution of this optimal density sequence and making local approximations, particle updates may be chosen which advance the state in an approximately optimal fashion. The process may be initialised by sampling from the transition density and weighting appropriately, in exactly the same manner as the prediction step of a bootstrap filter. (See section~\ref{sec:pf_basics}.)

Throughout this section, subscript $\rt$s are omitted for clarity on variables which vary with $\pt$. Particle superscripts are also omitted where appropriate.



\subsection{State Updates}

There is one class of models for which the OID has an analytic form, those which have a linear observation function and Gaussian transition and observation densities. The transition function need not be linear.
%
\begin{IEEEeqnarray}{rCl}
 p(\ls{\rt} | \ls{\rt-1}) & = & \normal{\ls{\rt}}{\transfun(\ls{\rt-1})}{\transcov} \nonumber \\
 p(\ob{\rt} | \ls{\rt})     & = & \normal{\ob{\rt}}{\obsmat \ls{\rt}}{\obscov}
\end{IEEEeqnarray}
%
For such models, the OID sequence~\eqref{eq:OID_sequence} is,
%
\begin{IEEEeqnarray}{rCl}
 \oiden{\pt}(\ls{\pt} | \ls{\rt-1}) & = & \mathcal{N}(\ls{\pt}|\lgoimean{\pt},\lgoicov{\pt}) \nonumber    ,
\end{IEEEeqnarray}
%
where
%
\begin{IEEEeqnarray}{rCl}
 \lgoicov{\pt} & = & \left[ \transcov^{-1} + \pt \obsmat^T \obscov^{-1} \obsmat \right]^{-1} \nonumber \\
 \lgoimean{\pt}    & = & \lgoicov{\pt} \left[ \transcov^{-1} \transfun(\ls{\rt-1}) + \pt \obsmat^T \obscov^{-1} \ob{\rt} \right] \nonumber     .
\end{IEEEeqnarray}
%
Since the OID may be sampled and the density evaluated, a composite proposal is redundant. However, the analytic formulas derived from this case may be used with other classes of models if local Gaussian approximations are made.

Consider the transition from $\pt_0$ to $\pt_1$. How should $\ls{\pt_0}$ be updated to get $\ls{\pt_1}$ such that it is distributed correctly? Since $\ls{\pt_0}|\ls{\rt-1}$ and $\ls{\pt_1}|\ls{\rt-1}$ are Gaussian random variables, they may be written,
%
\begin{IEEEeqnarray}{rCl}
 \ls{\pt_0} & = & \lgoimean{\pt_0} + \lgoicov{\pt_0}^{\frac{1}{2}} \stdnorm{\pt_0} \nonumber \\
 \ls{\pt_1} & = & \lgoimean{\pt_1} + \lgoicov{\pt_1}^{\frac{1}{2}} \stdnorm{\pt_1} \nonumber      ,
\end{IEEEeqnarray}
%
where $\stdnorm{\pt}$ is a standard normal random variable (zero mean, identity covariance matrix). One option is now simply to equate $\stdnorm{\pt_0}$ and $\stdnorm{\pt_1}$. However, it will be desirable later that if multiple particles have the same state at $\pt_0$, then they should move to different states at $\pt_1$. This may be achieved by adding a new random variable $\stdnorm{\Delta}$, which allows identical particles to `de-correlate'. Thus, to achieve the correct distribution of $\stdnorm{\pt_1}$, the following transformation is used,
%
\begin{IEEEeqnarray}{rCl}
 \stdnorm{\pt_1} & = & \lgdecayfunc(\pt_1-\pt_0) \stdnorm{\pt_0} + \sqrt{1-\lgdecayfunc(\pt_1-\pt_0)^2} \stdnorm{\Delta} \nonumber      ,
\end{IEEEeqnarray}
%
where $\lgdecayfunc(\pt) \in [0,1]$ is monotonically decreasing, and $\lgdecayfunc(\pt)\rightarrow0$ as $\pt\rightarrow0$. A suitable function is,
%
\begin{IEEEeqnarray}{rCl}
 \lgdecayfunc(\pt_1-\pt_0) & = & \exp\left\{-\half \lgexpsf (\pt_1-\pt_0)\right\} \nonumber       ,
\end{IEEEeqnarray}
%
where $\lgexpsf>0$. The resulting update formula for the state is,
%
\begin{IEEEeqnarray}{rCl}
 \ls{\pt_1} & = & \lgoimean{\pt_1} + \lgupdmeanmat{\pt_0,\pt_1}(\ls{\pt_0}-\lgoimean{\pt_0}) + \lgupdcov{\pt_0,\pt_1}^{\half} \stdnorm{\Delta} \nonumber \\
 \lgupdmeanmat{\pt_0,\pt_1} & = & \exp\left\{-\half\lgexpsf(\pt_1-\pt_0)\right\} \lgoicov{\pt_1}^{\half}\lgoicov{\pt_0}^{-\half} \nonumber \\
 \lgupdcov{\pt_0,\pt_1} & = & \left[1-\exp\left\{-\lgexpsf(\pt_1-\pt_0)\right\}\right]\lgoicov{\pt_1} \label{eq:state_update}      .
\end{IEEEeqnarray}
%
Provided $\lgexpsf\ne0$ this map has an associated incremental importance density,
%
\begin{IEEEeqnarray}{rCl}
 \impden(\ls{\pt_1} | \ls{\pt_0}) & = & \normal{\ls{\pt_1}}{\lgoimean{\pt_1} + \lgupdmeanmat{\pt_0,\pt_1}(\ls{\pt_0}-\lgoimean{\pt_0})}{ \lgupdcov{\pt_0,\pt_1}} \label{eq:incremental_importance_density}     .
\end{IEEEeqnarray}

For partially linear-Gaussian models, advancing the state through pseudo-time using \eqref{eq:state_update} generates a particle which is distributed exactly according to the current density in the OID sequence. For other models, such exact analytic updates do not exist. Instead, state updates may be conducted by forming a local Gaussian approximation to the OID sequence,
%
\begin{IEEEeqnarray}{rCl}
 \approxoiden{\pt}{\ls{\pt_0}}(\ls{\pt} | \ls{\rt-1}) & = & \mathcal{N}(\ls{\pt}|\lgoimeanapprox{\pt}{\ls{\pt_0}},\lgoicovapprox{\pt}{\ls{\pt_0}}) \nonumber \\
\end{IEEEeqnarray}
%
and then using \eqref{eq:state_update} again. Provided the pseudo-time increments are not too large, this process often leads to a more effective proposal than simpler approximations of the OID using a single Gaussian.



\subsection{Gaussian Approximations}

The CPPF relies on making a Gaussian approximations of the OID sequence at each point in pseudo-time by selecting $\lgoicovapprox{\pt}{\lsfixed}$ and $\lgoimeanapprox{\pt}{\lsfixed}$. This may be achieved using standard linearisation methods. The quality of the approximation does not affect the validity of the algorithm, since it will be corrected for in the assignment of a particle weight. However, better approximations will lead to higher effective sample sizes and higher quality estimates.

\subsubsection{Nonlinear Gaussian Models}

We begin with a class of models which are reasonably benign, and yet common in practice; those which have Gaussian transition and observation densities, but which are not linear,
%
\begin{IEEEeqnarray}{rCl}
 p(\ls{\rt} | \ls{\rt-1}) & = & \normal{\ls{\rt}}{\transfun(\ls{\rt-1})}{\transcov} \nonumber \\
 p(\ob{\rt} | \ls{\rt})   & = & \normal{\ob{\rt}}{\obsfun(\ls{\rt})}{\obscov}     .
\end{IEEEeqnarray}

For such models, the observation function may be linearised in the usual way, by truncating the Taylor series expansion,
%
\begin{IEEEeqnarray}{rCl}
 \obsfun(\ls{}) & \approx & \obsfun(\ls{}^*) + \underbrace{\left.\frac{\partial \obsfun}{\partial \ls{}}\right|_{\ls{}^*}}_{\obsmatapprox{\lsfixed}} (\ls{} - \ls{}^*)
\end{IEEEeqnarray}
%\begin{IEEEeqnarray}{rCl}
% p(\ob{\rt} | \ls{\rt}) & \approx & \normal{\ob{\rt}}{\obsfun(\lsfixed)+\obsmatapprox{\lsfixed}(\ls{\rt} - \lsfixed)}{\obscov} \nonumber      ,
%\end{IEEEeqnarray}
%
where $\lsfixed$ is the point around which the function is linearised, which will generally be the current state $\ls{\pt}$. An approximation of the OID sequence is then given by,
%
\begin{IEEEeqnarray}{rCl}
 \lgoicovapprox{\pt}{\lsfixed}  & = & \left[ \transcov^{-1} + \pt \obsmatapprox{\lsfixed}^T \obscov^{-1} \obsmatapprox{\lsfixed} \right]^{-1} \nonumber \\
 \lgoimeanapprox{\pt}{\lsfixed} & = & \lsfixed + \lgoicovapprox{\pt}{\lsfixed} \left[ \transcov^{-1} (\transfun(\ls{\rt-1})-\lsfixed) + \pt \obsmatapprox{\lsfixed}^T \obscov^{-1} (\ob{\rt}-\obsfun(\lsfixed)) \right] \label{eg:linearised_Gaussian_approx}     .
\end{IEEEeqnarray}

\subsubsection{Non-Gaussian Models}

If either model density is not Gaussian, a more generic approximation may be used. The Taylor series of the log-density of a Gaussian distribution is a quadratic function, so an arbitrary density may be approximated with a Gaussian by truncating to the first two terms. Applying this to both transition and observation densities we obtain,
%
\begin{IEEEeqnarray}{rCl}
 p(\ls{\rt} | \ls{\rt-1}) & \approx & \normal{\ls{\rt}}{ \transmeanapprox{\lsfixed} }{ \transcovapprox{\lsfixed} } \nonumber \\
 p(\ob{\rt} | \ls{\rt})   & \approx & \normal{ \obapprox{\lsfixed} }{ \ls{\rt} }{ \obscovapprox{\lsfixed} } \nonumber \\
 \transcovapprox{\lsfixed} & = & \left[\frac{\partial^2 \logtrans}{\partial \ls{}^2}_{\lsfixed}\right]^{-1} \nonumber \\
 \transmeanapprox{\lsfixed} & = & \lsfixed + \transcovapprox{\lsfixed} \frac{\partial \logtrans}{\partial \ls{}}_{\lsfixed} \nonumber \\
 \obscovapprox{\lsfixed} & = & \left[\frac{\partial^2 \logobs}{\partial \ls{}^2}_{\lsfixed}\right]^{-1} \nonumber \\
 \obapprox{\lsfixed} & = & \lsfixed + \obscovapprox{\lsfixed} \frac{\partial \logobs}{\partial \ls{}}_{\lsfixed} \nonumber       ,
\end{IEEEeqnarray}
%
where
%
\begin{IEEEeqnarray}{rCl}
 L(\ls{}) & = & \log\left(p(\ls{\rt}|\ls{\rt-1})\right) \nonumber \\
 M(\ls{}) & = & \log\left(p(\ob{\rt}|\ls{\rt})\right) \nonumber      .
\end{IEEEeqnarray}
%
The resulting approximation of the OID sequence is,
%
\begin{IEEEeqnarray}{rCl}
 \lgoicovapprox{\pt}{\lsfixed}  & = & \left[ \transcovapprox{\lsfixed}^{-1} + \pt \obscovapprox{\lsfixed}^{-1} \right]^{-1} \nonumber \\
 \lgoimeanapprox{\pt}{\lsfixed} & = & \lsfixed + \lgoicovapprox{\pt}{\lsfixed} \left[ \transcovapprox{\lsfixed}^{-1} \transmeanapprox{\lsfixed} + \pt \obscovapprox{\lsfixed}^{-1} \obapprox{\lsfixed} \right] \label{eg:general_Gaussian_approx}     .
\end{IEEEeqnarray}

Similar approximations have been used in, for example, \citep{Doucet2000a,Pitt1999} for selecting a single Gaussian importance density.

This generic approximation will not work if the log-densities do not have a negative curvature (i.e. a negative-definite Hessian) at $\lsfixed$. Furthermore, if the curvature in any direction is close to zero then the approximation can be very poor. {\meta Add a diagram to illustrate.} If such a failure occurs then various heuristics may be used to enforce the curvature condition. For example, one option is to perform an eigen-decomposition of the Hessian matrixes and replace any positive (or small negative) eigenvalues with a negative constant (e.g. the prior variance in that direction).



\subsection{Weight Updates}

Next we consider how to correctly weight particles generated using the composite proposal method. Two cases are considered separately, $\lgexpsf=0$ and $\lgexpsf\ne0$.

\subsubsection{Stochastic updates}

When $\lgexpsf\ne0$, particles are advanced through pseudo-time using a stochastic mechanism. Suppose a particle $\{\ls{1:\rt-1},\ls{\pt_0}\}$ with weight $\pw{\pt_0}$ distributed as $\augfiltden{\pt_0}$ already exists. This could be updated to $\pt_1$ by sampling $\ls{\pt_1}$ from the incremental importance density \eqref{eq:incremental_importance_density} and then discarding the old state $\ls{\pt_0}$. The distribution of the new particle would then be,
%
\begin{IEEEeqnarray}{rCl}
 \impden(\ls{1:\rt-1}, \ls{\pt_1}) & = & \int \augfiltden{\pt_0}(\ls{1:\rt-1},\ls{\pt_0}) \impden(\ls{\pt_1}|\ls{\pt_0}) d\ls{\pt_0} \nonumber      .
\end{IEEEeqnarray}
%
In the case of partially linear Gaussian models this may be calculated analytically, and is equal to the current density in the OID sequence. For other models, when approximations are used, this integral is not tractable because of the dependence of the approximations on the current state. Instead, tractable weight updates require the use of the extended target distribution method popularised by \citep{DelMoral2006}.

Instead of marginalising, the old state is added to the target distribution by extending it with an artificial conditional density,
%
\begin{IEEEeqnarray}{rCl}
 \augfiltden{\pt_1}(\ls{1:\rt-1},\ls{\pt_1}) \artden(\ls{\pt_0} | \ls{\pt_1}) \nonumber      .
\end{IEEEeqnarray}
%
After sampling, the distribution of the augmented particle $\{\ls{1:\rt-1},\ls{\pt_0},\ls{\pt_0}\}$ is now,
%
\begin{IEEEeqnarray}{rCl}
 \impden(\ls{1:\rt-1}, \ls{\pt_1}) & = & \augfiltden{\pt_0}(\ls{1:\rt-1},\ls{\pt_0}) \impden(\ls{\pt_1}|\ls{\pt_0}) \nonumber      .
\end{IEEEeqnarray}

The resulting incremental weight update formula is then,
%
\begin{IEEEeqnarray}{rCl}
 \pw{\pt_1} & \propto & \pw{\pt_0} \times \frac{ p(\ob{\rt} | \ls{\pt_1})^{\pt_1} p(\ls{\pt_1} | \ls{\rt-1}) }{ p(\ob{\rt} | \ls{\pt_0})^{\pt_0} p(\ls{\pt_0} | \ls{\rt-1}) } \times \frac{\artden(\ls{\pt_0} | \ls{\pt_1})}{\impden(\ls{\pt_1}|\ls{\pt_0})} \nonumber       .
\end{IEEEeqnarray}
%
In \citep{DelMoral2006}, the optimal form for the artificial density is shown to be,
%
\begin{IEEEeqnarray}{rCl}
 \artden(\ls{\pt_0} | \ls{\pt_1}) & = & \frac{ \approxoiden{\pt_0}{\ls{\pt_0}}(\ls{1:\rt-1}, \ls{\pt_0}) \impden(\ls{\pt_1} | \ls{\pt_0}) }{ \int \approxoiden{\pt_0}{\ls{\pt_0}}(\ls{1:\rt-1}, \ls{\pt_0}) \impden(\ls{\pt_1} | \ls{\pt_0}) d\ls{\pt_0} } \nonumber \\
 & = & \frac{ \approxoiden{\pt_0}{\ls{\pt_0}}(\ls{1:\rt-1}, \ls{\pt_0}) \impden(\ls{\pt_1} | \ls{\pt_0}) }{ \approxoiden{\pt_1}{\ls{\pt_0}}(\ls{1:\rt-1}, \ls{\pt_1}) } \label{eq:optimal_artificial_density}     .
\end{IEEEeqnarray}
%
Using this, the weight update for the CPPF becomes,
%
\begin{IEEEeqnarray}{rCl}
 \pw{\pt_1} & \propto & \pw{\pt_0} \times \frac{ p(\ob{\rt} | \ls{\pt_1})^{\pt_1} p(\ls{\pt_1} | \ls{\rt-1}) }{ p(\ob{\rt} | \ls{\pt_0})^{\pt_0} p(\ls{\pt_0} | \ls{\rt-1}) } \times \frac{\mathcal{N}(\ls{\pt_0}|\lgoimeanapprox{\pt_0}{\ls{\pt_0}},\lgoicovapprox{\pt_0}{\ls{\pt_0}})} {\mathcal{N}(\ls{\pt_1}|\lgoimeanapprox{\pt_1}{\ls{\pt_0}},\lgoicovapprox{\pt_1}{\ls{\pt_0}})} \label{eq:CPPF_stochastic_weight_update}       .
\end{IEEEeqnarray}



\subsubsection{Deterministic Updates}

The previous results are not strictly valid when $\lgexpsf=0$. In this case, the state update is completely deterministic, and the incremental importance density is not defined. Again suppose we have a particle $\{\ls{1:\rt-1},\ls{\pt_0}\}$ with weight $\pw{\pt_0}$ distributed as $\augfiltden{\pt_0}$. If a new state $\ls{\pt_1}$ is generated using \eqref{eq:state_update} and the old state $\ls{\pt_0}$ discarded, then the distribution of the resulting particle may be determined using the standard change of variable approach for a probability density,
%
\begin{IEEEeqnarray}{rCl}
 \impden(\ls{1:\rt-1},\ls{\pt_1} | \ls{\rt-1}) & = & \impden(\ls{1:\rt-1},\ls{\pt_0} | \ls{\rt-1}) \times \magdet{\frac{\partial \ls{\pt_{0}}}{\partial \ls{\pt_1}}}  \nonumber  .
\end{IEEEeqnarray}
%
The Jacobian for the state update \eqref{eq:state_update} is,
%
\begin{IEEEeqnarray}{rCl}
 \magdet{\frac{\partial \ls{\pt_{1}}}{\partial \ls{\pt_0}}} & = & \magdet{ \lgupdmeanmat{\pt_0,\pt_1} } \nonumber \\
 & = & \sqrt{\frac{\magdet{\lgoicovapprox{\pt_1}{\ls{\pt_0}}}}{\magdet{\lgoicovapprox{\pt_0}{\ls{\pt_0}}}}} \nonumber      .
\end{IEEEeqnarray}
%
Hence, the weight update is,
%
\begin{IEEEeqnarray}{rCl}
 \pw{\pt_1} & \propto & \pw{\pt_0} \times \frac{ p(\ob{\rt} | \ls{\pt_1})^{\pt_1} p(\ls{\pt_1} | \ls{\rt-1}) }{ p(\ob{\rt} | \ls{\pt_0})^{\pt_0} p(\ls{\pt_0} | \ls{\rt-1}) } \times \sqrt{\frac{\magdet{\lgoicovapprox{\pt_1}{\ls{\pt_0}}}}{\magdet{\lgoicovapprox{\pt_0}{\ls{\pt_0}}}}} \label{eq:CPPF_deterministic_weight_update}       .
\end{IEEEeqnarray}
%
It may easily be shown that as $\lgexpsf\rightarrow0$, \eqref{eq:CPPF_stochastic_weight_update} is equal to \eqref{eq:CPPF_deterministic_weight_update}.







\subsection{Adaptive Step Sizes}

An important consideration for the CPPF is how the sizes of the pseudo-time steps are chosen. State updates are calculated using local Gaussian approximations of the OID, so the best-case algorithm would take infinitesimal steps and use the appropriate approximation at each point on a continuous trajectory. (Note that such an algorithm still would not lead to proposals exactly from the OID, because it is still locally approximated.) In practice, the number of steps needs to be kept low, to minimise the computational burden. In some instances, it may be sufficient to use a fixed step size, or a predetermined time grid. However, an adaptive scheme is preferable for greatest efficiency.

\subsubsection{Local Error Estimates}

A measure is required which estimates the local ``error'' introduced by using finite rather than infinitesimal step sizes. For this, the following lemma will be useful.
%
\begin{lemma}\label{lem:state_SDE}
A state evolving according to the best-case algorithm obeys the following stochastic differential equation,
\begin{IEEEeqnarray}{rCl}
 d\ls{\pt} & = & \sdedrift{\pt|\ls{\pt_0}}(\ls{\pt}) d\pt + \sdediffuse{\pt|\ls{\pt_0}} d\lginfbm{\pt}      ,
\end{IEEEeqnarray}
%
where,
\begin{IEEEeqnarray}{rCl}
 \sdedrift{\pt|\ls{\pt_0}}(\ls{\pt}) & = & \lgoicov{\pt} \obsmat^T \obscov^{-1} \left( (\ob{\rt}-\obsmat\lgoimean{\pt}) - \half \obsmat (\ls{\pt}-\lgoimean{\pt}) \right) - \half \lgexpsf (\ls{\pt}-\lgoimean{\pt}) \nonumber \\
 \sdediffuse{\pt|\ls{\pt_0}}         & = & \lgexpsf^{\half} \lgoicovapprox{\pt}{\ls{\pt_0}}^{\half} \nonumber      ,
\end{IEEEeqnarray}
%
and $\lginfbm{\pt}$ is a standard Brownian motion process. See appendix~\ref{app:state_SDE} for proof.
\end{lemma}

The state error due to finite step sizes is then,
%
\begin{IEEEeqnarray}{rCl}
 \lserror{\pt_1}{\pt_0} & = & \int_{\pt_0}^{\pt_1} \left[\sdedrift{l|\ls{\pt_0}}(\ls{l}) - \sdedrift{l|\ls{l}}(\ls{l})\right] dl + \int_{\pt_0}^{\pt_1} \left[\sdediffuse{l|\ls{\pt_0}} - \sdediffuse{l|\ls{l}}\right] d\lginfbm{l} \nonumber      .
\end{IEEEeqnarray}
%
The drift and diffusion terms are correct at $\pt_0$, so for a crude approximation of this error, replace the integrands with half their final value. Then, using the fact that $\int_{\pt_0}^{\pt_1}d\lginfbm{l}=(\pt_1-\pt_0)^{\half} \stdnorm{\Delta}$, a rough but easily calculated estimate of the local error associated with this finite step is,
%
\begin{IEEEeqnarray}{rCl}
 \widehat{\lserror{\pt_1}{\pt_0}} & = & \half \left[\sdedrift{\pt_1|\ls{\pt_0}}(\ls{\pt_1}) - \sdedrift{\pt_1|\ls{\pt_1}}(\ls{\pt_1})\right] (\pt_1-\pt_0) \nonumber \\
 &   & \qquad \qquad + \: \half \left[\sdediffuse{\pt_1|\ls{\pt_0}}(\ls{\pt_1}) - \sdediffuse{\pt_1|\ls{\pt_1}}(\ls{\pt_1})\right] \stdnorm{\Delta} (\pt_1-\pt_0)^{\half}     .
\end{IEEEeqnarray}

\subsubsection{Step Size Control}

Using the estimate of local error we can now borrow a step size control mechanism directly from well established numerical integration algorithms for solving differential equations. One method found to be effective in practice, inspired by \citep{Shampine1997}, is to find the Euclidean norm of the state error estimate and then scale the step size according to,
%
\begin{IEEEeqnarray}{rCl}
 \Delta\pt_{\text{new}} & = & \Delta\pt_{\text{old}} \times a \left(\frac{\magdet{ \widehat{\lserror{\pt_1}{\pt_0}} }}{ e_{\text{tol}} } \right)^b \nonumber      ,
\end{IEEEeqnarray}
%
where $a$, $b$ and $e_{\text{tol}}$ are fixed parameters: $e_{\text{tol}}$ is the tolerance for the local error whereas $a$ and $b$ determine the response in the step size to deviations of the error estimate away from $e_{\text{tol}}$.



\subsection{Algorithm Summary}

The distinguishing novel components of the CPPF are the incremental state and weight update formulas for advancing the particles through pseudo-time. Algorithm~\ref{alg:general_CPPF} summarises the CPPF.

\begin{algorithm} \label{alg:general_CPPF}
\begin{algorithmic}[1]
  \FOR{$\rt=1,2,\dots$}
    \FOR{$i=1,\dots,N_F$}
      \IF{$\rt>1$}% and $\ess{\rt-1}$ less than threshold}
        \STATE Select an ancestor, $a_i=j$, with probability $\naw{\rt-1}\pss{j}$
        \STATE Calculate predictive particle weight, $\predpw{\rt}\pss{i} = \npw{\rt-1}\pss{a_i} / \naw{\rt-1}\pss{a_i}$.
      \ELSE
        \STATE Initialise weights, $\predpw{\rt}\pss{i} = 1$.
      \ENDIF
      \STATE Initialise pseudo-time, $\pt=0$.
      \STATE Initialise state by sampling from the transition/prior density, $\ls{\rt,0}\pss{i} \sim p(\ls{\rt} | \ls{\rt-1}\pss{\anc{i}})$ or $\ls{\rt,0}\pss{i} \sim p(\ls{\rt})$.
      \STATE Initialise weight, $\pw{\rt,0}\pss{i} = \predpw{\rt}\pss{i}$.
      \WHILE{$\pt<1$}
        \STATE Increment pseudo-time, $\pt \leftarrow \pt+\dpt$, using a fixed or adaptive method.
        \STATE Update state $\ls{\rt,\pt}\pss{i}$ using \eqref{eq:state_update}.
        \STATE Update weight $\pw{\rt,\pt}\pss{i}$ using \eqref{eq:CPPF_deterministic_weight_update} or \eqref{eq:CPPF_stochastic_weight_update}.
      \ENDWHILE
      \STATE Finalise, $\ls{\rt}\pss{i} = \ls{\rt,1}\pss{i}$, $\pw{\rt}\pss{i} = \pw{\rt,1}\pss{i}$.
    \ENDFOR
    \STATE Normalise weights, $\npw{\rt} = \pw{\rt}\pss{i} / \sum_j \pw{\rt}\pss{j}$ .
  \ENDFOR
\end{algorithmic}
\caption{Composite Proposal Particle Filter}
\end{algorithm}

The CPPF bears a resemblance to annealed particle filter algorithms \citep{Godsill2001b,Deutscher2000,Gall2007}, in its use of a stretch of pseudo-time between the predictive and filtering densities, yet it is quite distinct. The CPPF allows each particle to be handled independently, using different pseudo-time step sizes for each, and allows potentially deterministic state trajectories. In contrast, an annealed particle filter moves all the particles simultaneously through pseudo-time via a consistent set of bridging densities, interspersed with resampling and MCMC steps.



\section{Extensions to the CPPF}

\subsection{Composite Proposals for a Subset of Variables}

Some latent state variables may not be amenable to using a composite proposal, in particular discrete variables such as indicators. These may be handled by sampling them first from an ordinary importance distribution and then leaving them unchanged throughout the composite proposal process.

\subsection{Scale Mixtures of Normals}

Approximating a general density function by a Gaussian such as with \eqref{eg:general_Gaussian_approx} is crude, and often requires some heuristic adjustments to achieve good performance of the CPPF. A different method may be used for densities (either transition or observation) which may be represented as scale mixtures of normals,
%
\begin{IEEEeqnarray}{rCl}
 p(z) & = & \int \normal{z}{m}{\frac{1}{\mix{}}P} p(\mix{}) d\mix{}     .
\end{IEEEeqnarray}
%
For example, if $p(\mix{})$ is a chi-squared distribution (with $\dof$ degrees of freedom), then $p(z)$ is a student-t distribution (with $\dof$ degrees of freedom). With $1$ degree of freedom, this becomes a Cauchy distribution. Alpha-stable distributions may also be represented \citep{Godsill1999}.

By extending the target distribution to include the auxiliary mixing variable $\mix{}$, the density becomes conditionally Gaussian. There are now two options: either sample a single value of $\mix{}$ for each particle for the entire update, or sample a new value $\mix{\pt}$ for each step of the composite proposal. For the latter case, we use the following target density sequence,
%
\begin{IEEEeqnarray}{rCl}
 \augfiltden{\pt}(\ls{1:\rt-1}, \ls{\pt}, \mix{\pt}) & = & \frac{ p(\ob{\rt} | \ls{\rt,\pt}, \mix{\pt})^{\pt} p(\ls{\rt,\pt} | \ls{\rt-1}, \mix{\pt}) p(\mix{\pt}) p(\ls{1:\rt-1}|\ob{1:\rt-1}) }{ \augfiltnorm{\pt} } \label{eq:SMiN_filtering_sequence}      .
\end{IEEEeqnarray}
%
For the update from $\pt_0$ to $\pt_1$, a value of $\mix{\pt_1}$ is first sampled from a new importance density $\impden(\mix{\pt})$, after which the state update is carried out as before with the new value of $\mix{\pt_1}$, using \eqref{eq:state_update} (and \eqref{eg:linearised_Gaussian_approx} if required). The weight update formula with this modification is,
%
\begin{IEEEeqnarray}{rCl}
 \pw{\pt_1} & \propto & \pw{\pt_0} \times \frac{ p(\ob{\rt} | \ls{\pt_1}, \mix{\pt_1})^{\pt_1} p(\ls{\pt_1} | \ls{\rt-1}, \mix{\pt_1}) p(\mix{\pt_1}) }{ p(\ob{\rt} | \ls{\pt_0}, \mix{\pt_0})^{\pt_0} p(\ls{\pt_0} | \ls{\rt-1}, \mix{\pt_0}) p(\mix{\pt_0}) } \times \frac{\mathcal{N}(\ls{\pt_0}|\lgoimeanapprox{\pt_0}{\ls{\pt_0}},\lgoicovapprox{\pt_0}{\ls{\pt_0}}) \impden(\mix{\pt_1})} {\mathcal{N}(\ls{\pt_1}|\lgoimeanapprox{\pt_1}{\ls{\pt_0}},\lgoicovapprox{\pt_1}{\ls{\pt_0}}) \impden(\mix{\pt_0})} \nonumber       .
\end{IEEEeqnarray}

The simplest choice for $\impden(\mix{\pt})$ is to use the prior $p(\mix{\pt})$. This is very simple to implement, simplifies the weight formula, and has been shown to be effective in simulations. In addition, it can be used with alpha-stable distributions when the density function cannot be analytically evaluated. The optimal choice is to use the marginal conditional posterior,
%
\begin{IEEEeqnarray}{rCl}
 \impden(\mix{\pt}) & \propto & \int p(\ob{\rt} | \ls{\pt_1}, \mix{\pt_1})^{\pt_1} p(\ls{\pt_1} | \ls{\rt-1}, \mix{\pt_1}) p(\mix{\pt_1}) d\ls{\pt} \nonumber      ,
\end{IEEEeqnarray}
%
but in general it will be far to expensive to sample from this distribution.

The effect of sampling the mixing variable $\mix{\pt}$ is that for each step through psuedo-time, the state is updated according to a different conditionally-Gaussian model. Intuitively, this ``blend'' gives us an approximation to the update corresponding to the true density.



\subsection{Annealed CPPF}

The CPPF samples states from an approximation to the optimal importance density. However, even sampling the OID exactly would be no guarantee of a high effective sample size. The weight corresponding to a particles sampled from the OID is,
%
\begin{IEEEeqnarray}{rCl}
 p(\ob{\rt} | \ls{\rt-1}\pss{j})     ,
\end{IEEEeqnarray}
%
and the variance of a set of such weights can still be high, particularly if the true value of $\ls{\rt}$ or $\ob{\rt}$ is improbable given its modelled distribution. Many of the $t-1$ particles are simply in the wrong places given the information conveyed by the new observation, and will have low weights however good the importance density.

An effective method for addressing this problem is to use bridging densities \citep{Godsill2001b} or annealing \citep{Deutscher2000,Gall2007}. A full set of particles are sampled targeting \eqref{eq:filtering_sequence} for some fixed point in pseudo-time $\pt$. A selection step is then conducted before advancing the particles to the next point in pseudo-time. MCMC steps are added between selection and importance sampling to replenish the particle diversity.

The effect of using annealing is that unpromising low-weight particles are discarded early on, while those in high probability regions are replicated. The price of this improvement is that dependency is introduced between the particles, and the diversity of the particle histories (i.e. the number of unique histories $\ls{1:\rt-1}$) is reduced.

Annealing may be almost trivially integrated with the CPPF, since the particles are already moving through pseudo-time. A fixed grid of times is chosen at which to carry out a selection step. Each composite proposal is then stopped at these time by truncating the final step size, so that a full set of particles is collected, distributed as $\augfiltden{\pt}$. A selection step is then conducted, leaving a set of equally-weighted particles with the same distribution, and the composite proposal continues for each.

Provided $\lgexpsf>0$, the particles will automatically de-correlate between selection times, so no MCMC is necessary. Larger values of $\lgexpsf$ will reduce the dependence between multiple copies of a selected particle, but they will also require smaller step sizes to be used. Hence there exists a trade-off between the quality of the particle approximation and the computational load.

{\meta Diagram, algorithm pseudo-code or something to bulk this up a bit.}



\section{Numerical Simulations}

Numerical simulations have been conducted to demonstrate the efficacy of the CPPF. The purpose of the composite proposal method is to provide a more efficient particle filter for challenging nonlinear models. Simple Gaussian approximations work poorly, because the posterior filtering distributions of such models can assume weird and wonderful shapes. Unfortunately, this makes the assessment of particle filter performance a serious challenge. Statistics such as root-mean-square error (RMSE) and normalised estimation error squared (NEES), based on only the first two moments of the distribution, may be misleading. {\meta Diagram.}

Our primary indicator of performance will be the average effective sample size. RMSE values are also included, primarily to identify cases when algorithms ``lose track'' entirely. They should be treated with caution.

Effective sample sizes are measured before resampling, and may only be fairly compared between algorithms which propose particle states independently. In other words, it would not be appropriate to compare the CPPF to an annealed particle filter, although we will compare the CPPF with intermediate resampling to an annealed particle filter. The main contender for the CPPF will be a particle filter using the importance density suggested by \citet{Doucet2000a}, a Gaussian formed by truncating the Taylor series of the log-density of the OID around a local maximum. We refer to this algorithm as the optimal Gaussian importance particle filter (OGIPF)

\subsection{A Illustrative Problem}

The CPPF was tested on a modified form of the model used by \citet{Mihaylova2011}, which is a multivariate extension of the nonlinear benchmark model \citep{}. {\meta Cite something.} The model densities are Gaussian with transition and observation functions,
%
\begin{IEEEeqnarray}{rCl}
 \transfun(\ls{\rt-1}) & = & \half \ls{\rt-1} + 25 \frac{ \sum_d \ls{\rt-1,d} }{ 1 + \left(\sum_d \ls{\rt-1,d}\right)^2 } + 8 \cos(1.2 \rt) \nonumber \\
 \obsfun(\ls{\rt})_d   & = & \alpha \left( \ls{\rt,2d-1}^2 + \ls{\rt,2d}^2 \right) \nonumber      ,
\end{IEEEeqnarray}
%
where $\ls{\rt,d}$ and $\obsfun(\ls{\rt})_d$ are the $d$th components of the state vector and observation function respectively. The covariance matrixes for the transition and observation densities were set at $\transcov = 100 \times I$ and $\obscov = I$. A $10$-dimensional state and a $5$-dimensional observation were used.

This model is particularly challenging because the observations give us information only about the magnitudes of a set of sub-vectors of the state. Information about the corresponding bearing is only available via the transition model. Consequently, the region of high posterior probability corresponds to a ``thin'' section of the space bounding a hyper-sphere. A Gaussian density is a very poor approximation of this region. {\meta Reference to a diagram.} Particle filters using extended or unscented Kalman-type importance densities fail immediately on this model.

The following particle filters were tested on the multivariate benchmark model:
\begin{itemize}
  \item A bootstrap particle filter (BPF)
  \item An optimal Gaussian importance particle filter (OGIPF).
  \item A deterministic composite proposal particle filter with $\lgexpsf=0$ (D-CPPF).
  \item A stochastic composite proposal particle filter with $\lgexpsf=0.1$ (S-CPPF).
\end{itemize}

The number of particles used by each algorithm was selected so that the running times were roughly equal. The CPPFs both use the adaptive step size method. Figure~\ref{} shows the motion of the particles from the stochastic CPPF on a typical frame, and the awkward shape of the posterior mode. Figure~\ref{} shows the effective sample sizes of the particle approximations over a typical run. Table~\ref{} shows the average ESSs and RMSEs for each algorithm over 100 simulated data sets, each of 100 time steps.



\subsection{A Difficult Tracking Problem}

Now to a more applied problem. We consider tracking a small aircraft over a mapped landscape. Time of flight and doppler measurements from a radio transmitter on the aircraft provide accurate measurements of range $\rng{\rt}$, and range rate $\rng{\rt}$, but only a low resolution measurement of bearing $\bng{\rt}$. In addition, accurate measurements are made of the height above the ground $\hei{\rt}$. The profile of the terrain (i.e. the height of the ground above a datum at each point) has been mapped.

At $\rt$, the latent state for our model is,
%
\begin{IEEEeqnarray}{rCl}
 \ls{\rt} & = & \begin{bmatrix} \pos{\rt} \\ \vel{\rt} \end{bmatrix} \nonumber      ,
\end{IEEEeqnarray}
%
where $\pos{\rt}$ and $\vel{\rt}$ are the $3$-dimensional position and velocity of the aircraft respectively, and the observation is,
%
\begin{IEEEeqnarray}{rCl}
 \ob{\rt} & = & \begin{bmatrix} \bng{\rt} \\ \rng{\rt} \\ \hei{\rt} \\ \rngrt{\rt} \end{bmatrix}       .
\end{IEEEeqnarray}
%
The observation function is described by the following equations,
%
\begin{IEEEeqnarray}{rCl}
 \bng{\rt}   & = & \arctan\left(\frac{\pos{\rt,1}}{\pos{\rt,2}}\right) \nonumber \\
 \rng{\rt}   & = & \sqrt{ \pos{\rt,1}^2 + \pos{\rt,3}^2 + \pos{\rt,3}^2 } \nonumber \\
 \hei{\rt}   & = & \pos{\rt,3} - \terrain( \pos{\rt,1}, \pos{\rt,2} ) \nonumber \\
 \rngrt{\rt} & = & \frac{ \pos{\rt}\cdot\vel{\rt} }{ \rng{\rt} } \nonumber      ,
\end{IEEEeqnarray}
%
where $\terrain( \pos{\rt,1}, \pos{\rt,2} )$ is the terrain height at the corresponding horizontal coordinates. The four measurements are independent and the respective variances are $\frac{\pi}{9}^2$, $0.1^2$, $0.1^2$, $0.1^2$.

Two linear transition models have been used, both based on a near-constant velocity model, one with a Gaussian density and one with a Student-t density with $\dof = 3$ degrees of freedom,
%
\begin{IEEEeqnarray}{rCl}
 p_1(\ls{\rt} | \ls{\rt-1}) & = & \normal{\ls{\rt}}{\transmat\ls{\rt-1}}{\transcov} \nonumber \\
 p_2(\ls{\rt} | \ls{\rt-1}) & = & \studentt{\ls{\rt}}{\transmat\ls{\rt-1}}{\transcov}{\dof} \nonumber      ,
\end{IEEEeqnarray}
%
\begin{IEEEeqnarray}{rCl}
 \transmat & = & \begin{bmatrix} I & I \\ 0 & I \end{bmatrix} \nonumber \\
 \transcov & = & 10 \begin{bmatrix} \frac{1}{3} I & \frac{1}{2} I \\ \frac{1}{2} I &\ I \end{bmatrix} \nonumber \\
\end{IEEEeqnarray}

The accurate measurements of range, range rate and height constrain the region of high posterior probability to lie on a $3$ dimensional subspace, which can take some unusual shapes (see figure~\ref{}{\meta Add ref to a diagram}). This again means that particle filters using simple Gaussian importance densities do not perform well. For the simulations presented here, the terrain profile was modelled as a mixture of randomly-generated Gaussian blobs. An example is shown in figure~\ref{}.

The following particle filters were tested on the multivariate benchmark model:
\begin{itemize}
  \item A bootstrap particle filter (BPF).
  \item A particle filter using an EKF-type importance density (EPF).
  \item A particle filter using a UKF-type importance density (UPF).
  \item An optimal Gaussian importance particle filter (OGIPF).
  \item A deterministic composite proposal particle filter with $\lgexpsf=0$ (D-CPPF).
  \item A stochastic composite proposal particle filter with $\lgexpsf=0.1$ (S-CPPF).
\end{itemize}

In addition an annealed particle filter and a CPPF with intermediate resampling were also tested. The number of particles used by each algorithm was selected so that the running times were roughly equal. For the student-t transition density, the CPPFs use the scale mixture of normals method.

Figure~\ref{} shows the motion of the particles from the stochastic CPPF on a typical frame, and the awkward shape of the posterior mode. Figure~\ref{} shows the effective sample sizes of the particle approximations over a typical run with the Gaussian transition density.

Table~\ref{} shows the average ESSs and RMSEs for each algorithm over 100 simulated data sets, each of 100 time steps using the Gaussian transition density. The same is shown for the student-t transition density in table~\ref{}.



\subsection{A Heartbeat Inference Problem}

As a final example, we consider the problem of detecting heartbeats in a vibration signal. Measurements from an accelerometer are first partitioned into segments believed to contain a heartbeat, and a particle filter is then used to infer its properties. The $t$th heartbeat is modelled parametrically as the product of a Gaussian envelope with amplitude $\amp{\rt}$ and width $\wid{\rt}$, and a sine wave carrier with frequency $\freq{\rt}$ and relative phase $\pha{\rt}$. The time shift of the centre of the heartbeat within the measurement is $\del{\rt}$, and the sensor exhibits a D.C. bias $\bias{\rt}$ which varies slowly over time. The resulting observation function is highly nonlinear, with the $d$th component given by,
%
\begin{IEEEeqnarray}{rCl}
 \obsfun(\ls{\rt})_d & = & \amp{\rt} \exp\left\{ -\frac{ (T\,d - \del{\rt})^2 }{ 2\wid{\rt}^2 } \right\} \sin\left( \freq{\rt}(T\,d - \del{\rt}) + \pha{\rt} \right) + \bias{\rt} \nonumber      ,
\end{IEEEeqnarray}
%
where $T$ is the sampling period of the sensor. Each observation consists of $50$ time samples and the observation density is modelled as a Gaussian with a covariance matrix $0.2^2 I$. An example heartbeat simulated from this model is shown in figure~\ref{}.

The latent state is,
%
\begin{IEEEeqnarray}{rCl}
 \ls{\rt} & = & \begin{bmatrix} \amp{\rt} & \wid{\rt} & \del{\rt} & \freq{\rt} & \pha{\rt} & \bias{\rt} \end{bmatrix}^T      .
\end{IEEEeqnarray}
%
The transition density is factorised into independent terms, with $\freq{\rt}$, $\pha{\rt}$ and $\bias{\rt}$ evolving according to a Gaussian random walk, and $\wid{\rt}$ according to a geometric random walk (i.e. with a log-normal density), while $\del{\rt}$ and $\amp{\rt}$ are gamma distributed with no dependence on their past values.
%
%\begin{IEEEeqnarray}{rCl}
% p(\amp{\rt} )                & = & \gammaden{\amp{\rt}-0.5}{10}{0.05} \nonumber \\
% p(\wid{\rt} | \wid{\rt-1})   & = & \lognormal{\wid{\rt}}{}{} \nonumber \\
% p(\del{\rt})                 & = &   \nonumber \\
% p(\freq{\rt} | \freq{\rt-1}) & = &   \nonumber \\
% p(\pha{\rt} | \pha{\rt-1})   & = &   \nonumber \\
% p(\bias{\rt} | \bias{\rt-1}) & = &   \nonumber      ,
%\end{IEEEeqnarray}
%%
%where {\meta PARAMETERS}.

The complications introduced in the posterior distribution by the highly multi-modal observation density mean that simple Gaussian importance densities do not work very well. Particle filters using extended or unscented Kalman-type importance densities fail immediately on this model.

In order to use a CPPF with this model, the OID density sequence must be approximated with Gaussians. This is achieved by linearising the observation model around the current state \eqref{eg:linearised_Gaussian_approx} and using the log-density Taylor series truncation method for the prior \eqref{eg:general_Gaussian_approx}. The variance terms in this latter approximation are limited to be no greater than the prior variances.

The following particle filters were tested on the multivariate benchmark model:
\begin{itemize}
  \item A bootstrap particle filter (BPF)
  \item An optimal Gaussian importance particle filter (OGIPF).
  \item A deterministic composite proposal particle filter with $\lgexpsf=0$ (D-CPPF).
\end{itemize}

The number of particles used by each algorithm was selected so that the running times were roughly equal. The CPPFs both use the adaptive step size method. Figure~\ref{} shows the motion of the particles from the stochastic CPPF on a typical frame. Figure~\ref{} shows the effective sample sizes of the particle approximations over a typical run. Table~\ref{} shows the average ESSs and RMSEs for each algorithm over 100 simulated data sets, each of 100 time steps.



\section{Conclusions}

\section{THINGS TO DO}
\begin{itemize}
  \item Add simulation results
  \item Add number-of-steps
\end{itemize}



\appendix

\section{Proof of Lemma~\ref{lem:state_SDE}: A Stochastic Differential Equation For State Evolution} \label{app:state_SDE}

For clarity, write $\lgoimean{\pt}$ instead of $\lgoimeanapprox{\pt}{\ls{\pt_0}}$ and $\lgoicov{\pt}$ instead of $\lgoicovapprox{\pt}{\ls{\pt_0}}$ throughout this proof.

From \eqref{eq:state_update}, the state update for a short interval of pseudo-time is,
%
\begin{IEEEeqnarray}{rCl}
 \ls{\pt+\dpt} & = & \lgoimean{\pt+\dpt} + \lgupdmeanmat{\pt,\pt+\dpt}(\ls{\pt}-\lgoimean{\pt}) + \lgupdcov{\pt,\pt+\dpt}^{\half} \stdnorm{\Delta} \nonumber \\
 \lgupdmeanmat{\pt,\pt+\dpt} & = & \exp\left\{-\half\lgexpsf\dpt\right\} \lgoicov{\pt+\dpt}^{\half}\lgoicov{\pt}^{-\half} \nonumber \\
 \lgupdcov{\pt,\pt+\dpt} & = & \left[1-\exp\left\{-\lgexpsf\dpt\right\}\right]\lgoicov{\pt+\dpt} \nonumber         .
\end{IEEEeqnarray}

For small intervals,
%
\begin{IEEEeqnarray}{rCl}
 \exp\left\{-\half\lgexpsf\dpt\right\} & = & 1 - \half\lgexpsf\dpt + \bigo{\dpt^2} \nonumber \\
 \lgoimean{\pt+\dpt}                   & = & \lgoimean{\pt} + \dpt \frac{\partial \lgoimean{\pt}}{\partial \pt} + \bigo{\dpt^2} \nonumber \\
 \lgoicov{\pt+\dpt}                    & = & \lgoicov{\pt}  + \dpt \frac{\partial \lgoicov{\pt} }{\partial \pt} + \bigo{\dpt^2} \nonumber \\
                                       & = & \left[ I + \dpt \frac{\partial \lgoicov{\pt} }{\partial \pt} \lgoicov{\pt}^{-1} \right] \lgoicov{\pt} + \bigo{\dpt^2} \nonumber \\
 \lgoicov{\pt+\dpt}^{\half}            & = & \left[ I + \half \dpt \frac{\partial \lgoicov{\pt} }{\partial \pt} \lgoicov{\pt}^{-1} \right] \lgoicov{\pt}^{\half} + \bigo{\dpt^2} \nonumber      .
\end{IEEEeqnarray}
%
Hence,
%
\begin{IEEEeqnarray}{rCl}
 \ls{\pt+\dpt} - \ls{\pt} & = & \lgoimean{\pt} + \dpt \frac{\partial \lgoimean{\pt}}{\partial \pt} + \left(1 - \half\lgexpsf\dpt + \bigo{\dpt^2}\right) \left( I + \half \dpt \frac{\partial \lgoicov{\pt} }{\partial \pt} \lgoicov{\pt}^{-1} \right)(\ls{\pt}-\lgoimean{\pt}) \nonumber \\
 &   & \qquad \qquad + \: \dpt^{\half} \lgexpsf^{\half} \left( I + \half \dpt \frac{\partial \lgoicov{\pt} }{\partial \pt} \lgoicov{\pt}^{-1} \right) \lgoicov{\pt}^{\half} \stdnorm{\Delta} + \bigo{\dpt^2} - \ls{\pt} \nonumber \\
  & = & \left[ \frac{\partial \lgoimean{\pt}}{\partial \pt} + \half \left( \frac{\partial \lgoicov{\pt} }{\partial \pt} \lgoicov{\pt}^{-1} - \lgexpsf I \right) (\ls{\pt}-\lgoimean{\pt}) \right] \dpt + \lgexpsf^{\half} \lgoicov{\pt}^{\half} \dpt^{\half} \stdnorm{\Delta} + \bigo{\dpt^2} \nonumber      .
\end{IEEEeqnarray}
%
In the limit $\dpt\rightarrow0$, this becomes a stochastic differential equation,
%
\begin{IEEEeqnarray}{rCl}
 d\ls{\pt} & = & \left[ \frac{\partial \lgoimean{\pt}}{\partial \pt} + \half \left( \frac{\partial \lgoicov{\pt} }{\partial \pt} \lgoicov{\pt}^{-1} - \lgexpsf I \right) (\ls{\pt}-\lgoimean{\pt}) \right] d\pt + \lgexpsf^{\half} \lgoicov{\pt}^{\half} d\lginfbm{\pt}      ,
\end{IEEEeqnarray}
%
where $\lginfbm{\pt}$ is a standard Brownian motion process.

When the OID is, or is approximated by, a Gaussian with moments,
%
\begin{IEEEeqnarray}{rCl}
 \lgoicov{\pt} & = & \left[ \transcov^{-1} + \pt \obsmat^T \obscov^{-1} \obsmat \right]^{-1} \nonumber \\
 \lgoimean{\pt}    & = & \lgoicov{\pt} \left[ \transcov^{-1} \transfun(\ls{\rt-1}) + \pt \obsmat^T \obscov^{-1} \ob{\rt} \right] \nonumber     .
\end{IEEEeqnarray}
%
then
%
\begin{IEEEeqnarray}{rCl}
 \frac{\partial \lgoicov{\pt} }{\partial \pt} & = & -\lgoicov{\pt} \obsmat^T \obscov^{-1} \obsmat \lgoicov{\pt} \nonumber \\
 \frac{\partial \lgoimean{\pt}}{\partial \pt} & = & \lgoicov{\pt} \obsmat^T \obscov^{-1}(\ob{\rt}-\obsmat\lgoimean{\pt}) \nonumber       ,
\end{IEEEeqnarray}
%
and the state evolution stochastic differential equation is,
%
\begin{IEEEeqnarray}{rCl}
 d\ls{\pt} & = & \left[ \lgoicov{\pt} \obsmat^T \obscov^{-1} \left( (\ob{\rt}-\obsmat\lgoimean{\pt}) - \half \obsmat (\ls{\pt}-\lgoimean{\pt}) \right) - \half \lgexpsf (\ls{\pt}-\lgoimean{\pt}) \right] d\pt + \lgexpsf^{\half} \lgoicov{\pt}^{\half} d\lginfbm{\pt} \nonumber       .
\end{IEEEeqnarray}



\bibliographystyle{chicago}
\bibliography{D:/pb404/Dropbox/PhD/OTbib}
%\bibliography{/home/pete/Dropbox/PhD/OTbib}

\end{document} 