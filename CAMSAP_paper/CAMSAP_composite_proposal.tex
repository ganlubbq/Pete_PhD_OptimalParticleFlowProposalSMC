\documentclass[conference]{IEEEtran}
%\documentclass[draftcls]{IEEEtran}

\usepackage{cite}
\usepackage{graphicx}
\usepackage[caption=false,font=footnotesize]{subfig}
\usepackage{color}
\usepackage{amsmath}
\usepackage{amsfonts}
\usepackage{bbm}
\usepackage{algorithmic}

\graphicspath{{figures/}}
%%% MACROS FOR MATHEMATICAL NOTATION IN COMPOSITE PROPOSAL PAPER %%%

% Functions and operators
\newcommand{\half}{\frac{1}{2}}                                 % Half
\newcommand{\expect}[1]{\mathbb{E}_{#1}}                        % Expectation
\newcommand{\variance}[1]{\mathbb{V}_{#1}}                      % Variance
\DeclareMathOperator{\trace}{Tr}                                % Trace
\newcommand{\magdet}[1]{\left| #1 \right|}         % Magnitude of the determinant
\newcommand{\indic}[1]{\mathbbm{1}_{#1}}                        % Indicator function
\newcommand{\normal}[3]{\mathcal{N}\left(#1\left|#2,#3\right.\right)}       % Normal density
\newcommand{\gammaden}[3]{\mathcal{\Gamma}\left(#1|#2,#3\right)}% Gamma density
\newcommand{\studentt}[4]{\mathcal{ST}\left(#1|#2,#3,#4\right)} % Student-t density
\newcommand{\bigo}[1]{\mathcal{O}\left({#1}\right)}             % Big O
\newcommand{\mhaccept}{\alpha}                                  % Metropolis-Hastings acceptance probability

% Basics
\newcommand{\rt}{t}                             % Real time
\newcommand{\pt}{\lambda}                       % Pseudo-time
\newcommand{\dpt}{\delta\lambda}                % A little bit of pseudo-time
\newcommand{\ls}[1]{x_{#1}}                     % Latent state
\newcommand{\ob}[1]{y_{#1}}                     % Observation
\newcommand{\mix}[1]{\xi_{#1}}                  % Mixing auxiliary variable
\newcommand{\els}[1]{u_{#1}}                    % Extra latent state

% Particle shizzle
\newcommand{\pss}[2][]{^{(#2)#1}}               % Particle superscript
\newcommand{\pw}[1]{w_{#1}}                     % Particle weight
\newcommand{\predpw}[1]{\hat{w}_{#1}}           % Predictive particle weight
\newcommand{\npw}[1]{\bar{w}_{#1}}              % Normalised particle weight
\newcommand{\naw}[1]{\bar{v}_{#1}}              % Normalised auxiliary weight
\newcommand{\anc}[1]{a_{#1}}                    % Particle ancestor

% Densities
\newcommand{\transden}{f}                       % Transition density
\newcommand{\obsden}{g}                         % Observation density
\newcommand{\impden}{q}                         % Importance density
\newcommand{\partden}{\eta}                     % Unweighted particle distribution
\newcommand{\artden}{\rho}                      % Artificial conditional density
\newcommand{\oiden}[1]{\pi_{#1}}                % Optimal importance density
\newcommand{\approxoiden}[2]{\hat{\pi}_{#1|#2}} % Approximation of the optimal importance density
\newcommand{\augfiltden}[1]{\tilde{\pi}_{#1}}   % Augmented filtering density
\newcommand{\oinorm}[1]{K_{#1}}                 % Normalising constant for the optimal importance density
\newcommand{\augfiltnorm}[1]{\tilde{K}_{#1}}    % Normalising constant for the augmented filtering density

% Numbers
\newcommand{\numpart}{N_F}                      % Number of filter particles
\newcommand{\ess}[1]{N_{E,#1}}                  % Effective sample size

% Models
\newcommand{\transfun}{\phi}                    % Transition function
\newcommand{\obsfun}{h}                         % Observation function
\newcommand{\transcov}{Q}                       % Transition covariance
\newcommand{\obscov}{R}                         % Observation covariance
\newcommand{\transmat}{F}                       % Linear transition matrix
\newcommand{\obsmat}{H}                         % Linear observation matrix
\newcommand{\transmean}{m}                      % Mean of the transition density (i.e. f(x_{t-1}))
\newcommand{\dof}{\nu}                          % Degrees of freedom of something student-t-ish

% Linear Gaussian things
\newcommand{\lgoimean}[1]{\mu_{#1}}             % Linear Gaussian optimal importance density mean
\newcommand{\lgoicov}[1]{\Sigma_{#1}}           % Linear Gaussian optimal importance density covariance
\newcommand{\stdnorm}[1]{z_{#1}}                % Standard normal R.V.

% Gaussian Transformation
\newcommand{\lgdecayfunc}{a}                    % Linear Gaussian decay function for OID transformation
\newcommand{\lgexpsf}{\gamma}                   % Linear Gaussian exponential scale factor for OID transformation
\newcommand{\lgupdmeanmat}[1]{\Gamma_{#1}}      % Mean mapping matrix for the OID transformation
\newcommand{\lgupdcov}[1]{\Omega_{#1}}          % Covariance matrix for the OID transformation
\newcommand{\lginfbm}[1]{\epsilon_{#1}}         % Brownian motion for the infinitesimal form of the OID transformation

% Linear Gaussian approximations
%\newcommand{\lgoimeanapprox}[2]{\hat{\mu}_{#1}(#2)}     % Mean of the Gaussian approximation to the OID at time #1 and state #2
%\newcommand{\lgoicovapprox}[2]{\hat{\Sigma}_{#1}(#2)}   % Covariance of the Gaussian approximation to the OID at time #1 and state #2
\newcommand{\lgoimeanapprox}[2]{\hat{\mu}_{#1|#2}}      % Mean of the Gaussian approximation to the OID at time #1 and state #2
\newcommand{\lgoicovapprox}[2]{\hat{\Sigma}_{#1|#2}}    % Covariance of the Gaussian approximation to the OID at time #1 and state #2
\newcommand{\obsmatapprox}[1]{\hat{\obsmat}_{#1}}       % Linear observation matrix formed by differentiation of the observation function
\newcommand{\transmeanapprox}[1]{\hat{\transmean}_{#1}} % Approximate transition mean
\newcommand{\transcovapprox}[1]{\hat{\transcov}_{#1}}   % Approximate transition covariance
\newcommand{\obapprox}[1]{\hat{y}_{#1}}                 % Approximate observation mean
\newcommand{\obscovapprox}[1]{\hat{\obscov}_{#1}}       % Approximate observation covariance
\newcommand{\lsfixed}{\ls{}^*}                          % Latent state around which we linearise
\newcommand{\logtrans}{L}                               % Log of the transition density
\newcommand{\logobs}{M}                                 % Log of the observation density

% State evolution SDE
\newcommand{\oudrift}[1]{A_{#1}}                % General O-U process drift term
\newcommand{\oudiffuse}[1]{B_{#1}}              % General O-U process diffusion term
\newcommand{\sdedrift}[1]{\zeta_{#1}}           % Drift
\newcommand{\lserror}[2]{e_{#1|#2}}             % State error due to finite sampling

% Particle flow
\newcommand{\flowbm}[1]{\epsilon_{#1}}          % Particle flow Brownian motion
\newcommand{\flowdrift}[1]{\zeta_{#1}}          % Particle flow drift
\newcommand{\flowdiffuse}[1]{\eta_{#1}}         % Particle flow diffusion
\newcommand{\flowcov}[1]{D_{#1}}                % Particle flow "Covariance" matrix
\newcommand{\flowtd}{\alpha}                    % Flow dervation transition density
\newcommand{\flowod}{\beta}                     % Flow derivation observation density

% Simulation models - tracking
\newcommand{\pos}[1]{p_{#1}}           % Position
\newcommand{\vel}[1]{v_{#1}}           % Velocity
\newcommand{\bng}[1]{\theta_{#1}}               % Bearing
\newcommand{\rng}[1]{r_{#1}}                    % Range
\newcommand{\hei}[1]{h_{#1}}                    % Height
\newcommand{\rngrt}[1]{s_{#1}}                  % Range rate
\newcommand{\terrain}{T}                        % Terrain height

% Simulation models - heartbeats
\newcommand{\amp}[1]{A_{#1}}                    % Amplitude
\newcommand{\wid}[1]{W_{#1}}                    % Width
\newcommand{\del}[1]{\tau_{#1}}                 % Delay
\newcommand{\freq}[1]{\omega_{#1}}              % Width
\newcommand{\pha}[1]{\psi_{#1}}                 % Phase
\newcommand{\bias}[1]{B_{#1}}                   % Bias


\hyphenation{}

\begin{document}

\title{Particle Filtering with Composite Gaussian Approximations to the Optimal Importance Density}
\author{\IEEEauthorblockN{Pete Bunch}
\IEEEauthorblockA{Signal Processing and Communications Lab.\\
Cambridge University Engineering Dept.\\
Cambridge, UK,\\
Email: pb404@cam.ac.uk}
%\and
%\IEEEauthorblockN{Simon Godsill}
%\IEEEauthorblockA{Signal Processing and Communications Lab.\\
%Cambridge University Engineering Dept.\\
%Cambridge, UK,\\
%Email: sjg30@cam.ac.uk}
}

\maketitle

\begin{abstract}
The abstract goes here.
\end{abstract}


\IEEEpeerreviewmaketitle



\section{Introduction}

Particle filtering

\subsection{Subsection Heading Here}



\subsubsection{Subsubsection Heading Here}



% An example of a floating figure using the graphicx package.
% Note that \label must occur AFTER (or within) \caption.
% For figures, \caption should occur after the \includegraphics.
% Note that IEEEtran v1.7 and later has special internal code that
% is designed to preserve the operation of \label within \caption
% even when the captionsoff option is in effect. However, because
% of issues like this, it may be the safest practice to put all your
% \label just after \caption rather than within \caption{}.
%
% Reminder: the "draftcls" or "draftclsnofoot", not "draft", class
% option should be used if it is desired that the figures are to be
% displayed while in draft mode.
%
%\begin{figure}[!t]
%\centering
%\includegraphics[width=2.5in]{myfigure}
% where an .eps filename suffix will be assumed under latex,
% and a .pdf suffix will be assumed for pdflatex; or what has been declared
% via \DeclareGraphicsExtensions.
%\caption{Simulation Results}
%\label{fig_sim}
%\end{figure}

% Note that IEEE typically puts floats only at the top, even when this
% results in a large percentage of a column being occupied by floats.


% An example of a double column floating figure using two subfigures.
% (The subfig.sty package must be loaded for this to work.)
% The subfigure \label commands are set within each subfloat command, the
% \label for the overall figure must come after \caption.
% \hfil must be used as a separator to get equal spacing.
% The subfigure.sty package works much the same way, except \subfigure is
% used instead of \subfloat.
%
%\begin{figure*}[!t]
%\centerline{\subfloat[Case I]\includegraphics[width=2.5in]{subfigcase1}%
%\label{fig_first_case}}
%\hfil
%\subfloat[Case II]{\includegraphics[width=2.5in]{subfigcase2}%
%\label{fig_second_case}}}
%\caption{Simulation results}
%\label{fig_sim}
%\end{figure*}
%
% Note that often IEEE papers with subfigures do not employ subfigure
% captions (using the optional argument to \subfloat), but instead will
% reference/describe all of them (a), (b), etc., within the main caption.


% An example of a floating table. Note that, for IEEE style tables, the
% \caption command should come BEFORE the table. Table text will default to
% \footnotesize as IEEE normally uses this smaller font for tables.
% The \label must come after \caption as always.
%
%\begin{table}[!t]
%% increase table row spacing, adjust to taste
%\renewcommand{\arraystretch}{1.3}
% if using array.sty, it might be a good idea to tweak the value of
% \extrarowheight as needed to properly center the text within the cells
%\caption{An Example of a Table}
%\label{table_example}
%\centering
%% Some packages, such as MDW tools, offer better commands for making tables
%% than the plain LaTeX2e tabular which is used here.
%\begin{tabular}{|c||c|}
%\hline
%One & Two\\
%\hline
%Three & Four\\
%\hline
%\end{tabular}
%\end{table}


% Note that IEEE does not put floats in the very first column - or typically
% anywhere on the first page for that matter. Also, in-text middle ("here")
% positioning is not used. Most IEEE journals/conferences use top floats
% exclusively. Note that, LaTeX2e, unlike IEEE journals/conferences, places
% footnotes above bottom floats. This can be corrected via the \fnbelowfloat
% command of the stfloats package.



\section{Conclusion}
The conclusion goes here.




% conference papers do not normally have an appendix


% use section* for acknowledgement
\section*{Acknowledgment}


The authors would like to thank...


% trigger a \newpage just before the given reference
% number - used to balance the columns on the last page
% adjust value as needed - may need to be readjusted if
% the document is modified later
%\IEEEtriggeratref{8}
% The "triggered" command can be changed if desired:
%\IEEEtriggercmd{\enlargethispage{-5in}}

\bibliographystyle{IEEEtran}
\bibliography{D:/pb404/Dropbox/PhD/OTbib}

\end{document}


