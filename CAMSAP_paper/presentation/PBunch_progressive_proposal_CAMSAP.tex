\documentclass{beamer}
\usetheme{Boadilla}

\usepackage{amsmath}
\usepackage{IEEEtrantools}
\usepackage{subfig}
\usepackage{color}

% General things
\newcommand{\half}{\frac{1}{2}}
\newcommand{\pdv}[2]{\frac{\partial #1}{\partial #2}}
\newcommand{\determ}[1]{\left| #1 \right|}

% Density families
\newcommand{\normalden}[3]{\mathcal{N}\left(#1\left|#2,#3\right.\right)}       % Normal density

% Basics
\newcommand{\ti}{n}                             % Time index
\newcommand{\pt}{\lambda}                       % Pseudo-time
\newcommand{\ls}[1]{x_{#1}}                     % Latent state
\newcommand{\ob}[1]{y_{#1}}                     % Observation
\newcommand{\stdnorm}[1]{z_{#1}}                % Standard normal R.V.

% Densities
\newcommand{\den}{p}                            % Secondary Bayesian density
\newcommand{\transden}{f}                       % Transition density
\newcommand{\obsden}{g}                         % Observation density
\newcommand{\impden}{q}                         % Importance density
\newcommand{\partden}{\eta}                     % Unweighted particle distribution
\newcommand{\artden}{\rho}                      % Artificial conditional density
\newcommand{\oiden}[1]{\pi_{#1}}                % Optimal importance density
\newcommand{\augfiltden}[1]{\tilde{\pi}_{#1}}   % Augmented filtering density


% Particle shizzle
\newcommand{\pss}[2][]{^{(#2)#1}}               % Particle superscript
\newcommand{\pw}[1]{w_{#1}}                     % Particle weight

% Models
\newcommand{\transfun}{\phi}                    % Transition function
\newcommand{\obsfun}{\psi}                      % Observation function
\newcommand{\transcov}{Q}                       % Transition covariance
\newcommand{\obscov}{R}                         % Observation covariance
\newcommand{\obsmat}{H}                         % Linear observation matrix 
\newcommand{\obsmatapprox}[1]{\hat{\obsmat}_{#1}} % Linear observation matrix formed by differentiation of the observation function

% Gaussian moments
\newcommand{\lsmn}[1]{m_{#1}}                   % Mean
\newcommand{\lsvr}[1]{P_{#1}}                   % Variance
\newcommand{\obmn}[1]{\mu_{#1}}
\newcommand{\obvr}[1]{\Sigma_{#1}}
\newcommand{\obcvr}[1]{C_{#1}}

% Simulation models - tracking
\newcommand{\pos}[1]{p_{#1}}                    % Position
\newcommand{\vel}[1]{v_{#1}}                    % Velocity
\newcommand{\bng}[1]{\theta_{#1}}               % Bearing
\newcommand{\rng}[1]{r_{#1}}                    % Range
\newcommand{\hei}[1]{h_{#1}}                    % Height
\newcommand{\rngrt}[1]{s_{#1}}                  % Range rate
\newcommand{\terrain}{T}                        % Terrain height
\graphicspath{{../../paper/figures/}{CAMSAP_plots/}}
\captionsetup[subfloat]{labelformat=empty}

\title[Progressive Proposals]{Particle Filtering with Progressive Gaussian Approximations to the Optimal Importance Density}
\author[P. Bunch \& S. Godsill]{Pete Bunch and Simon Godsill}
\institute[CUED SigProC]{Cambridge University Engineering Department\\ Signal Processing \& Communications Lab}
\date{17th December, 2012}

\begin{document}

\begin{frame}
 \titlepage
\end{frame}
\begin{frame}{The Plan}
 
\end{frame}


\begin{frame}{Particle Filtering}
\begin{IEEEeqnarray*}{rCcCl}
 \ls{\ti} & \sim & \transden(\ls{\ti} | \ls{\ti-1}) & = & \normalden{\ls{\ti}}{\transfun(\ls{\ti-1})}{\transcov} \\
 \ob{\ti} & \sim & \obsden(\ob{\ti} | \ls{\ti})     & = & \normalden{\ob{\ti}}{\obsfun(\ls{\ti})}{\obscov} \\
 \ls{1} & \sim & p(\ls{1})                          & = & \normalden{\ls{1}}{\lsmn{1}}{\lsvr{1}}      ,
\end{IEEEeqnarray*}
\pause
\begin{itemize}
 \item Particle filter approximates:
 \begin{IEEEeqnarray*}{rCl}
  \den(\ls{1:\ti}|\ob{1:\ti}) & \propto & \transden(\ls{\ti}|\ls{\ti-1}) \obsden(\ob{\ti}|\ls{\ti}) \den(\ls{1:\ti-1}|\ob{1:\ti-1})
 \end{IEEEeqnarray*}
 \item Select an ($\ti-1$) particle and sample a new state, $\ls{\ti}\pss{i} \sim \impden(\ls{\ti}|\ls{\ti-1}\pss{i})$.
 \item Update weight $\pw{\ti}\pss{i} = \frac{ \transden(\ls{\ti}\pss{i}|\ls{\ti-1}\pss{i}) \obsden(\ob{\ti}|\ls{\ti}\pss{i}) }{ \impden(\ls{\ti}\pss{i}|\ls{\ti-1}\pss{i}) }$.
\end{itemize}
\end{frame}
\begin{frame}{Importance Densities}
\pause Bootstrap filter
\begin{IEEEeqnarray*}{rCl}
 \impden(\ls{\ti}|\ls{\ti-1}) = \transden(\ls{\ti}|\ls{\ti-1})     .
\end{IEEEeqnarray*}
\pause Optimal Importance Density
\begin{IEEEeqnarray*}{rCl}
 \impden(\ls{\ti}|\ls{\ti-1}) = \frac{ \transden(\ls{\ti}|\ls{\ti-1}) \obsden(\ob{\ti}|\ls{\ti}) }{ \int \transden(\ls{\ti}|\ls{\ti-1}) \obsden(\ob{\ti}|\ls{\ti}) d\ls{\ti} }      .
\end{IEEEeqnarray*}
Approximate by linearisation or sigma-point approximation.
\end{frame}


\begin{frame}{An Example}
\begin{figure}
\centering
\subfloat[Prior]{\includegraphics[width=0.5\columnwidth]{path_ex_prior.pdf}}
\subfloat[Likelihood]{\includegraphics[width=0.5\columnwidth]{path_ex_lhood.pdf}}
\end{figure}
\end{frame}
\begin{frame}{An Example - Linearisation}
\begin{figure}
\centering
\subfloat[True Posterior]{\includegraphics[width=0.5\columnwidth]{path_ex_post_contour.pdf}}
\subfloat[Approximated by Linearisation]{\includegraphics[width=0.5\columnwidth]{path_ex_ekf.pdf}}
\end{figure}
\end{frame}
\begin{frame}{An Example - Unscented Transform}
\begin{figure}
\centering
\subfloat[True Posterior]{\includegraphics[width=0.5\columnwidth]{path_ex_post_contour.pdf}}
\subfloat[Approximated by Unscented Transform]{\includegraphics[width=0.5\columnwidth]{path_ex_ukf.pdf}}
\end{figure}
\end{frame}


\begin{frame}{Progressive Principle}
Introduce the observation gradually using a stretch of ``pseudo-time'', $\pt \in [0,1]$. \\
\vspace{1em}
Smooth sequence of target distributions:
 \begin{IEEEeqnarray*}{rCl}
 \augfiltden{\ti,\pt}(\ls{1:\ti-1}, \ls{\ti,\pt}) & = & \obsden(\ob{\ti} | \ls{\ti,\pt})^{\pt} \transden(\ls{\ti,\pt} | \ls{\ti-1}) p(\ls{1:\ti-1}|\ob{1:\ti-1})
\end{IEEEeqnarray*}
Smooth sequence of optimal importance densities:
\begin{IEEEeqnarray*}{rCl}
 \oiden{\ti,\pt}(\ls{\ti,\pt} | \ls{\ti-1}\pss{j}) & = & \obsden(\ob{\ti} | \ls{\ti,\pt})^{\pt} \transden(\ls{\ti,\pt} | \ls{\ti-1}\pss{j})
\end{IEEEeqnarray*}
\end{frame}
\begin{frame}{Partially Linear Gaussian Models}
\begin{IEEEeqnarray*}{rCl}
 \transden(\ls{\ti}|\ls{\ti-1}) & = & \normalden{\ls{\ti}}{\transfun(\ls{\ti-1})}{\transcov} \\
 \obsden(\ob{\ti}|\ls{\ti})     & = & \normalden{\ob{\ti}}{\obsmat \ls{\ti}}{\obscov}
\end{IEEEeqnarray*}
\vspace{1em}
Optimal importance density is analytically tractable.
\begin{IEEEeqnarray*}{rCl}
 \oiden{\pt}(\ls{\pt} | \ls{\ti-1}) & = & \normalden{\ls{\pt}}{\lsmn{\pt}}{\lsvr{\pt}}    ,
\end{IEEEeqnarray*}
\begin{IEEEeqnarray*}{rCl}
 \lsvr{\pt}  & = & \left[ \transcov^{-1} + \pt \obsmat^T \obscov^{-1} \obsmat \right]^{-1} \\
 \lsmn{\pt} & = & \lsvr{\pt} \left[ \transcov^{-1} \transfun(\ls{\ti-1}) + \pt \obsmat^T \obscov^{-1} \ob{\ti} \right]       .
\end{IEEEeqnarray*}
\end{frame}
\begin{frame}{Partially Linear Gaussian Models}
Any Gaussian random variable may be expressed as,
\begin{IEEEeqnarray*}{rCl}
 \ls{} & = & \lsmn{} + \lsvr{}^{\half} \stdnorm{} \\
 \stdnorm{} & \sim & \normalden{\stdnorm{}}{0}{I}      .
\end{IEEEeqnarray*}
Progress particle from $\pt_0$ to $\pt_1$ with a linear transformation.
\begin{IEEEeqnarray*}{rCl}
 \ls{\pt_1} & = & \lsmn{\pt_1} + \lsvr{\pt_1}^{\half}\lsvr{\pt_0}^{-\half}(\ls{\pt_0}-\lsmn{\pt_0})       .
\end{IEEEeqnarray*}
\end{frame}
\begin{frame}{Partially Linear Gaussian Models}
\begin{figure}
\centering
\includegraphics[width=0.5\columnwidth]{plg_oid_evolution.pdf}
\end{figure}
\end{frame}

\begin{frame}{Nonlinear Gaussian Models}
Approximate the optimal importance density with a Gaussian at each point in pseudo-time.
\begin{IEEEeqnarray*}{rCl}
 \oiden{\pt}(\ls{\pt} | \ls{\ti-1}) & \approx & \normalden{\ls{\pt}}{\lsmn{\pt}}{\lsvr{\pt}}
\end{IEEEeqnarray*}
\vspace{1em}
To update from $\pt_0$ to $\pt_1$,
\begin{IEEEeqnarray*}{rCl}
 \oiden{\pt_1}(\ls{}) & \propto & \oiden{\pt_0}(\ls{}) \obsden(\ob{\ti}|\ls{})^{\pt_1-\pt_0} \\
 & = & \normalden{\ls{}}{\lsmn{\pt_0}{\pt_0}}{\lsvr{\pt_0}} \normalden{\ob{\ti}}{\obsfun(\ls{})}{\obscov}^{\pt_1-\pt_0}      .
\end{IEEEeqnarray*}
\end{frame}
\begin{frame}{Nonlinear Gaussian Models}
Approximate with linearisation.
\begin{IEEEeqnarray*}{rCl}
 \obsmatapprox{\ls{\pt_0}} & = & \pdv{\obsfun}{\ls{}}{\ls{\pt_0}}
\end{IEEEeqnarray*}
\vspace{1em}
\begin{IEEEeqnarray*}{rCl}
 \obmn{\pt_0}  & = & \obsfun(\ls{\pt_0}) + \obsmatapprox{\ls{\pt_0}} ( \lsmn{\pt_0} - \ls{\pt_0} ) \nonumber \\
 \obvr{\pt_0}  & = & \obsmatapprox{\ls{\pt_0}} \lsvr{\pt_0} \obsmatapprox{\ls{\pt_0}}^T \\
 \obcvr{\pt_0} & = & \lsvr{\pt_0} \obsmatapprox{\ls{\pt_0}}^T \\
 \lsmn{\pt_1}  & = & \lsmn{\pt_0} + \obcvr{\pt_0} \left(\obvr{\pt_0}+ \frac{\obscov}{\pt_1-\pt_0}\right)^{-1} \left(\ob{\ti}-\obmn{\pt_0}\right)  \\
 \lsvr{\pt_1}  & = & \lsvr{\pt_0} - \obcvr{\pt_0} \left(\obvr{\pt_0}+ \frac{\obscov}{\pt_1-\pt_0}\right)^{-1} \obcvr{\pt_0}^T
\end{IEEEeqnarray*}
\end{frame}

\begin{frame}{Weight Evolution}
\begin{itemize}
 \item Existing particle at pseudo time $\pt_0$,
 \begin{IEEEeqnarray*}{rCl}
  \left\{ \ls{1:\ti-1}\pss{i}, \ls{\pt_0}\pss{i}, \pw{\pt_0}\pss{i} \right\} \sim \partden(\ls{1:\ti-1}, \ls{\pt_0})     .
 \end{IEEEeqnarray*}
 \item ... is replaced by new particle  at $\pt_1$,
 \begin{IEEEeqnarray*}{rCl}
  \left\{ \ls{1:\ti-1}\pss{i}, \ls{\pt_1}\pss{i}, \pw{\pt_1}\pss{i} \right\} \sim \partden(\ls{1:\ti-1}, \ls{\pt_1})     .
 \end{IEEEeqnarray*}
 \item Standard change of variables formula,
 \begin{IEEEeqnarray*}{rCl}
  \partden(\ls{1:\ti-1},\ls{\pt_1}) & = & \partden(\ls{1:\ti-1},\ls{\pt_0}) \times \determ{ \pdv{\ls{\pt_0}}{\ls{\pt_1}} } \nonumber      .
 \end{IEEEeqnarray*}
\end{itemize}
%\end{frame}
%\begin{frame}{Weight Evolution}
Hence weight update,
\begin{IEEEeqnarray*}{rCl}
 \pw{\pt_1} & = & \pw{\pt_0} \times \frac{ \obsden(\ob{\ti} | \ls{\pt_1})^{\pt_1} \transden(\ls{\pt_1} | \ls{\ti-1}) }{ \obsden(\ob{\ti} | \ls{\pt_0})^{\pt_0} \transden(\ls{\pt_0} | \ls{\ti-1}) } \times \sqrt{\frac{\determ{\lsvr{\pt_1}}}{\determ{\lsvr{\pt_0}}}}      .
\end{IEEEeqnarray*}
%\begin{IEEEeqnarray*}{rCl}
% \pw{\pt_1} & = & \frac{ \augfiltden{\pt_1}(\ls{1:\ti-1},\ls{\pt_1}) }{ \partden(\ls{1:\ti-1},\ls{\pt_1}) } \nonumber \\
% & = & \frac{ \augfiltden{\pt_0}(\ls{1:\ti-1},\ls{\pt_0}) }{ \partden_{\pt_0}(\ls{1:\ti-1},\ls{\pt_0}) } \times \frac{ \augfiltden{\pt_1}(\ls{1:\ti-1},\ls{\pt_1})}{ \augfiltden{\pt_0}(\ls{1:\ti-1},\ls{\pt_0}) } \times \determ{ \pdv{\ls{\pt_1}}{\ls{\pt_0}} } \nonumber \\
% & \propto & \pw{\pt_0} \times \frac{ \obsden(\ob{\ti} | \ls{\pt_1})^{\pt_1} \transden(\ls{\pt_1} | \ls{\ti-1}) }{ \obsden(\ob{\ti} | \ls{\pt_0})^{\pt_0} \transden(\ls{\pt_0} | \ls{\ti-1}) } \times \sqrt{\frac{\determ{\lsvr{\pt_1}}}{\determ{\lsvr{\pt_0}}}}      .
%\end{IEEEeqnarray*}
\end{frame}

\begin{frame}{Relationship to Other Methods}
Gradual introduction of likelihood underlies numerous existing particle methods,
\begin{itemize}
 \item LIST BRIDGING DISTRIBUTION, ANNEALING, TEMPERING, PROGRESSIVE CORRECTION PAPERS.
\end{itemize}
Distinguishing features of the new progressive proposal method:
\begin{itemize}
 \item Particles moved deterministically.
 \item Move one particle at a time --- no intermediate interaction steps needed.
 \item Adaptive step-size control for \emph{each particle}.
\end{itemize}
\end{frame}


\begin{frame}{Simulations - A Tracking Problem}
\begin{IEEEeqnarray*}{rCl}
 \ls{\ti} & = & \begin{bmatrix} \pos{\ti}^T & \vel{\ti}^T \end{bmatrix}^T
\end{IEEEeqnarray*}
Near-constant velocity model.
\begin{IEEEeqnarray*}{rCl}
 \ob{\ti} & = & \begin{bmatrix} \bng{\ti} & \rng{\ti} & \hei{\ti} & \rngrt{\ti} \end{bmatrix}^T       .
\end{IEEEeqnarray*}
\begin{IEEEeqnarray*}{rCl}
 \bng{\ti}   & = & \arctan\left(\frac{\pos{\ti,1}}{\pos{\ti,2}}\right)\\
 \rng{\ti}   & = & \sqrt{ \pos{\ti,1}^2 + \pos{\ti,3}^2 + \pos{\ti,3}^2 } \\
 \hei{\ti}   & = & \pos{\ti,3} - \terrain( \pos{\ti,1}, \pos{\ti,2} ) \\
 \rngrt{\ti} & = & \frac{ \pos{\ti}\cdot\vel{\ti} }{ \rng{\ti} }       ,
\end{IEEEeqnarray*}
\end{frame}
\begin{frame}{Simulations - Terrain Map}
\begin{IEEEeqnarray*}{rCl}
 \terrain( \pos{\ti,1}, \pos{\ti,2} )
\end{IEEEeqnarray*}
\begin{figure}
\centering
\includegraphics[width=0.5\columnwidth]{drone_terrain_map.pdf}
\end{figure}
\end{frame}
\begin{frame}{Simulations - Results}
\begin{figure}
\centering
\includegraphics[width=0.5\columnwidth]{drone_example_frame_deter.pdf}
\end{figure}
\end{frame}
\begin{frame}{Simulations - Results}
\begin{table}
\centering
\begin{tabular}{l||c|c|c}
Algorithm                                & $N_F$ & ESS  & RMSE \\
\hline
Bootstrap Proposal                       &  6000 &  1.0 & 78.6 \\
Unscented Kalman Proposal                &   460 &  2.4 & 70.2 \\
Gaussian Local Maximum Proposal          &    10 &  3.1 & 62.9 \\
Progressive Proposal                     &   180 & 56.4 & 22.3 \\
\end{tabular}
\end{table}
\end{frame}

\begin{frame}{Summary}
\begin{itemize}
 \item The progressive proposal method samples from an effective approximation of the optimal importance density.
 \item Particles sampled from the transition density, then moved deterministically with a series of approximately optimal steps.
 \item Lower errors and better sample sizes when filtering with challenging nonlinear models.
\end{itemize}
\end{frame}

\begin{frame}

\end{frame}


\end{document}